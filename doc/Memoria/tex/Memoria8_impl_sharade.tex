\documentclass[class=article, crop=false]{standalone}
\usepackage[utf8]{inputenc}
\usepackage[spanish]{babel}
\usepackage{minted}
\usepackage{xcolor}
\usepackage{stmaryrd}

\definecolor{bg}{rgb}{0.95,0.95,0.95}

\begin{document}

\section{Implementación de Sharade}

Toda la implementación se encuentra presente en el siguiente repositorio de GitHub:
https://github.com/ManuelVs/Sharade.

La implementación cuenta con 14 módulos y alrededor de mil líneas de código Haskell. La mayor
parte de las líneas de código corresponden a la traducción y el análisis de tipos. Este
relativamente pequeño tamaño resulta, en mi opción, engañoso para apreciar una gran
dificultad. La programación en Haskell, y especialmente la programación monádica, puede ser
extraordinariamente concisa y expresiva, requiriendo eso sí, una pericia técnica del
programador no trivial de adquirir. Por otra parte, también hay que destacar que la
implementación hace uso de una versión modificada de la biblioteca \verb`explicit-sharing` de
Fischer para hacerla funcionar con las versiones actuales de Haskell. Realizar esa
modificación nos ha exigido comprender en profundidad la biblioteca de Fischer.

La implementación de \verb`Sharade` se encuentra amparada por la licencia de software libre
\verb`MIT`. La versión modificada de la biblioteca de Fischer es completamente dominio
público, igual que la versión original.

\subsection{Sintaxis concreta de Sharade}\label{sec:sintaxis_concreta}

En la sección \ref{sec:sintaxis_abstracta} presentamos las diferentes construcciones
sintácticas consideradas en Sharade en un formato que podríamos considerar de sintaxis
abstracta. Damos aquí la sintaxis concreta  en forma de reglas gramaticales que pueden servir
ya de base directa a la construcción de un analizador sintáctico, que describimos en la
sección \ref{sec:analizador_sintactico}.

\begin{itemize}
  \item[Program] $\rightarrow$ \verb`{ FunDecl }`
  \item[FunDecl] $\rightarrow$ \verb`Identifier { Identifier } "=" Expression ";"`
  \item[Expression] $\rightarrow$ \verb`E0`
  \item[E0] $\rightarrow$ \verb`E1 "?" E0`
  \item[E0] $\rightarrow$ \verb`E1`
  \item[E1] $\rightarrow$ \verb`E2 "&&" E1 | E2 "||" E1`
  \item[E1] $\rightarrow$ \verb`E2`
  \item[E2] $\rightarrow$ \verb`E3 RelOperator E2`
  \item[E2] $\rightarrow$ \verb`E3`
  \item[E3] $\rightarrow$ \verb`E4 "+" E3 | E4 "-" E3`
  \item[E3] $\rightarrow$ \verb`E4`
  \item[E4] $\rightarrow$ \verb`E5 "*" E4 | E5 "/" E4`
  \item[E4] $\rightarrow$ \verb`E5`
  \item[E5] $\rightarrow$ \verb`Lambda | ChooseIn | LetIn | CaseExpr` \\
                          \verb`| FExp | Pair | "(" E0 ")"`
  \item[FExp] $\rightarrow$ \verb`Identifier { AExp }`
  \item[AExp] $\rightarrow$ \verb`LitExpr | Variable | "(" E0 ")"`
  \item[Lambda] $\rightarrow$ \verb`"\" Identifier "->" Expression`
  \item[ChooseIn] $\rightarrow$ \verb`"choose" Identifier "=" Expression "in" Expression`
  \item[LetIn] $\rightarrow$ \verb`"let" Identifier "=" Expression "in" Expression`
  \item[Pair] $\rightarrow$ \verb`"(" Expression "," Expression ")"`
  \item[CaseExpr] $\rightarrow$ \verb`"case" Expression "of" { CaseMatch }`
  \item[CaseMatch] $\rightarrow$ \verb`Pattern "->" Expression ";"`
  \item[Pattern] $\rightarrow$ \verb`Constructor { Identifier }`
  \item[Constructor] $\rightarrow$ \verb`UpperCase { AlphaNum }`
\end{itemize}

Un identificador válido es cualquier cadena de caracteres alfanumérica que empiece por un
carácter. El lenguaje también permite comentarios de una línea y multilínea, igual que
Haskell. Los comentarios multilínea empiezan con `\verb`{-`' y terminan con `\verb`-}`'. Los
comentarios de una sola línea empiezan por `\verb`--`'. La sintaxis no permite la creación de
nuevos operadores infijos por el usuario, al contrario que Haskell \cite{marlow2010haskell}.
La sintaxis está claramente influenciada por Haskell y Curry. En la sintaxis se incluyen
también las primitivas necesarias para expresar el no determinismo con compartición y sin
compartición. El operador \verb`?` expresa la elección de dos valores no deterministas y la
estructura \verb`choose ... in ...` expresa una nueva ligadura de una expresión a un
identificador que compartirá el valor en todas sus apariciones, cumpliendo la propiedad que
Francisco J. Fraguas muestra en `A simple rewrite notion for call-time choice semantics'
\cite{lopez2007simple}.

\subsection{Analizador sintáctico con parsec}\label{sec:analizador_sintactico}

Como ya hemos visto, la biblioteca \verb`parsec` permite especificar una gramática
L-atribuida aprovechando todas las características de Haskell. Un inconveniente que tiene es
que en análisis es descendente, por lo que a la hora de la implementación hay que transformar
la gramática para evitar la recursión a izquierdas. En mi implementación no he hecho un
cálculo de los directores, por lo que el análisis se basa en un backtracking LL0.

La implementación es bastante sencilla, pero hace un uso interesante de las características
de orden superior de Haskell. Tan solo quiero mostrar una pequeña parte de la
implementación para enseñar cómo he llevado a cabo la transformación de la gramática de
atributos para poder analizar descendentemente usando dichas características de orden
superior de Haskell.

\begin{minted}[bgcolor=bg]{haskell}
createExpr :: String -> Expr -> Expr -> Expr
createExpr op lexp rexp = (App (App (Var prefixNot) lexp) rexp)
  where prefixNot = "(" ++ op ++ ")"

expr' :: Int -> IParser Expr
expr' 0 = do
  le <- expr' 1
  cont <- expr'' 0
  return (cont le)

expr'' :: Int -> IParser (Expr -> Expr)
expr'' 0 = do
    reservedOp "?"
    rexp <- expr' 1
    cont <- expr'' 0
    return (\lexp -> createExpr "?" lexp (cont rexp))
  <%> return (\e -> e)
\end{minted}

Esta parte del código corresponde al análisis de las expresiones con un operador infijo de
prioridad cero, que es la máxima prioridad. En este caso es solo corresponde al operador
\verb`(?)`, origen del indeterminismo de mi lenguaje.

Como se puede ver, las definiciones auxiliares en vez de obtener un valor heredado con la
expresión, como se haría en un analizador descendente tradicional, devuelven una función que
espera dicha expresión. Es decir, las definiciones auxiliares analizan una función de
Haskell. Esto puede sonar raro, pero no lo es teniendo en cuenta que empiezan reconociendo un
operador infijo, por lo que la expresión que falta es la parte izquierda del operador. La
otra opción, la que viene después del operador \verb`<%>` es el caso base. En este caso se
devuelve la función identidad, análogo a devolver el valor heredado. La implementación de
esta parte se puede encontrar en el módulo \verb`Sharade.Parser`.

\subsection{Análisis de tipos}

Como se ha indicado varias veces, la implementación de Sharade procede por traducción a un
programa Haskell cuya ejecución recoge la funcionalidad esperada del programa original. La
traducción utiliza intensivamente la programación monádica, como se ha explicado en el
apartado \ref{sec:uso_directo}. El programa traducido, al ser Haskell, debe estar bien
tipado. En un primer momento se podría pensar en delegar en Haskell el análisis de tipos,
reportando su propio interpretador un error de tipos si fuese necesario. Pero debido a la
naturaleza de la traducción (al uso intensivo de la programación monádica), se incurre
fácilmente en ambigüedades de tipos, por lo que Haskell rechaza el programa con un caso
particular de error de tipos. Para evitar esta situación, es necesario realizar un análisis
de tipos sobre el programa en Sharade, con el clásico sistema de
Hindley-Milner \cite{hindley1969principal}. En el apartado en el que se explica la traducción
a Haskell indicaré cómo se transforman estos tipos inferidos a Haskell.

Veamos los siguientes ejemplos para entender las situaciones de ambigüedad de tipos y cómo
solucionarlas:

\begin{minted}[bgcolor=bg]{haskell}
import Control.Monad

zero = return 0

coin = (return 0) `mplus` (return 1)
\end{minted}

Si se intenta interpretar esto en Haskell dará un error, debido a la ambigüedad de tipos.
Haskell podría elegir una mónada cualquiera para analizar el tipo, pero eso quebraría el
supuesto de `mundo abierto'. Si se añadiese una nueva instancia que se ajustase mejor a la
situación, el código que has escrito podría dejar de funcionar o ni siquiera compilar. Veamos
el segundo ejemplo donde se ve más claro \cite{Lipovaca:2011:LYH:2018642}:

\begin{minted}[bgcolor=bg]{haskell}
main :: IO ()
main = putStrLn (show (yesno 12))

class YesNo a where
  yesno :: a -> Bool
  
instance YesNo [a] where
  yesno [] = False
  yesno _  = True

instance YesNo Int where
  yesno 0 = False
  yesno _ = True
\end{minted}

Este ejemplo no compilará. No sabe qué instancia coger para evaluar \verb`yesno 12`. Es
evidente, con solo estas líneas de código, que la única instancia que encaja es
\verb`YesNo Int`. Pero si en el futuro se añade la instancia \verb`YesNo Integer`, habría
ambigüedad e incluso podría cambiar el comportamiento esperado del programa. Esto es lo que
evita el supuesto del mundo abierto. Si no se sabe que puede haber otra instancia que encaje,
el sistema de tipos dará un error de ambigüedad.

Esto se puede solucionar permitiendo que las expresiones traducidas sean polimórficas, es
decir, permitiendo que se puedan evaluar en la mónada que se desee. Para conseguir esto es
necesario especificar directamente los tipos, por lo que es necesario un análisis de tipos de
mi lenguaje y traducirlos correctamente a Haskell. Veámoslo en el ejemplo:

\begin{minted}[bgcolor=bg]{haskell}
import Control.Monad

zero :: MonadPlus m => m Int
zero = return 0

coin :: MonadPlus m => m Int
coin = (return 0) `mplus` (return 1)
\end{minted}

De alguna manera se está indicando a Haskell que el programa es correcto evaluando dichas
expresiones en cualquier instancia de la clase \verb`MonadPlus`, como puede ser la
constructora de las listas o la mónada que usa Fischer.

\subsubsection{Traducción de tipos a Haskell}

Tras inferir los tipos de las funciones en Sharade, es necesario traducirlos a Haskell. Como
ya se ha dicho en anteriores ocasiones, cada tipo en Haskell debe ser monádico. En la
traducción existen dos excepciones, que son la traducción del tipo de las listas y del tipo
de las parejas.

El tipo de una función de Sharade \verb`f :: t` se traduce a Haskell como \\
\verb`f :: Sharing s => s (translate t)`. El tipo \verb`t` puede caer en uno de los
siguientes casos:

\begin{itemize}
  \item[-] Un tipo primitivo \verb`p` se traduce como \verb`s p`.
  \item[-] Una variable de tipos \verb`a` se traduce como \verb`s a`.
  \item[-] El tipo funcional, \verb`a -> b`, se traduce como \\
  \verb`s (translate a -> translate b)`.

  \item[-] El tipo de las listas, \verb`[a]`, se traduce como \\
  \verb`s (List s (translate' a))`.

  \item[-] El tipo de las parejas, \verb`(a, b)`, se traduce como \\
  \verb`s (Pair s (translate' a) (translate' b))`
\end{itemize}

La función de traducción \verb`translate'` se comporta como \verb`translate` pero sin poner
delante el tipo monádico en el primer nivel de recursión, llamando a \verb`translate`
inmediatamente.

Los tipos de las listas y de las parejas se traducen de esta manera debido a que estos tipos
en Haskell gestionan el no determinismo (y las mónadas) dentro de su implementación. Por
ejemplo, la cola de las listas es un valor indeterminista, lo que no se conseguiría si
simplemente fuese una lista de valores indeterministas (\verb`[s a]`). Esto tiene fuertes
y buenas repercusiones en el funcionamiento de la evaluación perezosa de Sharade.

\subsubsection{Inferencia de tipos Hindley-Milner}

La familia de tipos de Hindley-Milner tiene la fantástica propiedad de tener un simple
algoritmo para determinar los tipos de una sintaxis sin tipos. El algoritmo se basa en unas
simples reglas para obtener el tipo de una expresión. El elemento principal de este algoritmo
es la `unificación' de variables de tipos. Las reglas son las siguientes
\cite{write-you-a-haskell}\cite{hindley1969principal}\cite{milner1978theory}:
\begin{itemize}
  \item[-] T-Var: Si `\verb`x : a`' pertenece al ámbito de tipos, entonces del ámbito de
  tipos se puede deducir `\verb`x : a`'.

  \item[-] T-App: Si `\verb`e1 : t1 -> t2`' y `\verb`e2 : t1`' pertenecen al ámbito de tipos,
  entonces se puede deducir que \verb`e1 e2 : t2`.
  
  \item[-] T-Lam: Si con el ámbito de tipos y `\verb`x : t1`' se puede deducir
  `\verb`e : t2`', entonces `\verb`\x . e : t1 -> t2`'.
  
  \item[-] T-Gen: Si del ámbito de tipos se deduce `\verb`e : a`' y la variable de tipo
  `\verb`b`' no está presente en el ámbito, entonces `\verb`e : forall b . a`'.

  \item[-] T-Inst: Si del ámbito de tipos se deduce `\verb`e : a`' y \verb`b` es una
  instancia de `\verb`a`', entonces del ámbito de tipos también se deduce `\verb`e : b`'.
\end{itemize}

El ámbito de tipos es una colección de expresiones y sus tipos. La regla
\verb`T-Var` es básica. La regla \verb`T-App` caracteriza la aplicación de una expresión de
tipo funcional a otra expresión. Si la parte derecha es del tipo que la función espera,
entonces el tipo de la aplicación es el resultado del tipo funcional. \verb`T-Lam` especifica
el tipo de una función (expresión lambda). Una expresión \verb`(\x -> e)` tiene un tipo
tipo funcional, donde el primer argumento debe tener el mismo tipo de `\verb`x`' y el
resultado tiene el tipo de `\verb`e`'. \verb`T-Gen` expresa la generalización de un tipo. Por
ejemplo, decir \verb`a : Int` es equivalente a decir que \verb`a : forall b . Int`, ya que
`\verb`b`' no aparece en la parte derecha. \verb`T-Inst` caracteriza la instanciación. Por
ejemplo una posible instanciación de la función \verb`id : a -> a` es \verb`id : Int -> Int`.

La unificación de dos tipos es el proceso de inferir si estos dos tipos pueden confluir en
uno solo, creando una sustitución en el proceso. Por ejemplo, los tipos `\verb`a`' y
`\verb`Char`' se pueden unificar y dan como resultado la sustitución \verb`[a/Int]`. Otra
sustitución válida sería \verb`a -> a ~ Int -> Int : [a/Int]`, para la función identidad. Se
puede dar el caso que existan substituciones infinitas. Por ejemplo, si intentamos
\verb`a ~ a -> b`, tenemos la sustitución `\verb`[a/a->b]`'. Si aplicamos esa sustitución,
siempre obtendremos un tipo más grande, debido a que `\verb`a`' aparece en la parte derecha
de la sustitución. La única sustitución posible sería \verb`a/((..((a->b)->b)..)->b)`, pero
ni Haskell ni Curry ni Sharade tienen tipos infinitos. Debido a esto, la unificación tiene
una precondición conocida como `occurs-check'. Si se va a unificar `\verb`a`' con `\verb`b`',
`\verb`a`' no debe aparecer en `\verb`b`' (ni viceversa), ya que terminaría en una
sustitución infinita. Esta parte de la implementación se puede encontrar en el módulo
\verb`Sharade.Translator.Sharade`.

\subsection{Traducción a Haskell}\label{sec:traduccion}

La última etapa de la compilación es la generación de código Haskell. Para explicar la
traducción me basaré en la sintaxis abstracta presentada en \ref{sec:sintaxis_abstracta}. En
la traducción de cualquier expresión se cumple un invariante, las subexpresiones generadas
son expresiones Haskell monádicas, por lo tanto nuestra función de traducción siempre debe
generar una expresión monádica. Esto es necesario para tener presente en todo momento el
indeterminismo. De esta manera, las expresiones funcionales pasan a estar dentro de mónadas,
por lo que para traducir la aplicación de una función es necesario definir un operador
auxiliar que se encargue de aplicar la función monádica a un valor monádico:

\begin{minted}[bgcolor=bg]{haskell}
infixl 4 <#>
(<#>) :: (Sharing s) => s (s a -> s b) -> s a -> s b
f <#> a = f >>= (\f' -> f' a)
\end{minted}

Este operador recibe una función atrapada en una mónada (que recibe valores monádicos) y un
valor monádico. Aplica la función indeterminista (puede que sean varias) al valor
indeterminista (puede que sean varios) y devuelve el resultado. El resultado puede ser de un
tipo primitivo o ser un tipo funcional (encerrado en una mónada). Así, este operador se puede
aplicar varias veces.

Veamos caso a caso como funciona la traducción de una expresión. Me he ahorrado los
paréntesis que serían necesarios en Haskell para facilitar la lectura.

\begin{itemize}
  \item[-]{\makebox[1.2cm]{$l$\hfill}} $\Rightarrow$ Un literal se traduce como
  \verb`return l`
  
  \item[-]{\makebox[1.2cm]{$x$\hfill}} $\Rightarrow$ Dependiendo de la variable se traduce de
  una forma u otra. Por ejemplo, el operador \verb`(+)` se traduce como \verb`mAdd`, el
  operador \verb`(?)` como \verb`mPlus`. Un identificador simple se traduce sin modificación.
  
  \item[-]{\makebox[1.2cm]{$C$\hfill}} $\Rightarrow$ Una constructora es un caso especial de
  variable, existe una traducción para cada constructora disponible. Por ejemplo, \verb`Cons`
  se traduce por la función auxiliar \verb`cons`.

  \item[-]{\makebox[1.2cm]{$e1 \: e2$\hfill}} $\Rightarrow$ Ambas subexpresiones son
  monádicas, por lo tanto hay que usar el operador explicado antes,
  \verb`translate e1 <#> translate e2`.

  \item[-]{\makebox[1.2cm]{$\backslash x \rightarrow e$\hfill}} $\Rightarrow$ Una función
  lambda es una función normal, tiene que estar metida en un valor monádico. Por lo tanto, se
  usa la función \verb`return` sobre la traducción de la expresión:
  \verb`return (\x -> translate e)`.

  \item[-]{\makebox[1.2cm]{$e1 \:? \: e2$\hfill}}  $\Rightarrow$ Esta expresión cae en los
  casos anteriores, traducción de una variable y dos aplicaciones funcionales. Su función
  auxiliar es \verb`mPlus`.

  \item[-]{\makebox[1.2cm]{$(e1, \: e2)$\hfill}} $\Rightarrow$ Parejas. Igual que el
  anterior, se traduce una constructora y dos aplicaciones funcionales. Su constructora en
  Haskell es \verb`mPair`.

  \item[-]{\makebox[3.4cm]{$choose \: x = e1 \: in \: e2$\hfill}} $\Rightarrow$ Compartición
  de una variable o expresión. Hay que usar la primitiva \verb`share` sobre la expresión y
  añadir al ámbito la nueva variable. He decidido usar una expresión lambda para conseguir
  este objetivo y el operador \verb`(>>=)`. Así, la traducción se hace de la siguiente
  manera: \verb`share (translate e1) >>= (\x -> translate e2)`

  \item[-]{\makebox[3.4cm]{$let \: x = e1 \: in \: e2$\hfill}} $\Rightarrow$ Una expresión
  \verb`let` se traduce sin ninguna complicación, se puede usar la misma estructura de
  Haskell: \\
  \verb`let x = translate e1 in translate e2`.

  \item[-]{\makebox[3.4cm]{$e1 \: (infixOp) \: e2$\hfill}} $\Rightarrow$ Una expresión con un
  operador infijo. Como antes, debido a la representación interna de las operaciones infijas,
  se traduce como una variable y dos aplicaciones funcionales.

  \item[-]$case \: e \: of \: t1 \rightarrow e1 ; \ldots tn \rightarrow en ;$ $\Rightarrow$
  Una expresión \verb`case` es complicada de traducir, ya que hay un caso especial en el que
  Haskell no puede deducir los `tipos intermedios' de las expresiones, por culpa de
  ambigüedades en las clases de tipos, dando un error en tiempo de compilación. Para traducir
  esta expresión, uso una función auxiliar llamada \verb`translateMatch`, que traduce cada
  uno de los casos del \verb`case`. Así, la expresión se traduce como \\
  \verb`translate e >>= (\pec -> case pec of {`\\
  \verb`translateMatch (t1 -> e1); ...; translateMatch (tn -> en);`\\
  \verb`_ -> mzero;})` \\
  Veamos cual es el funcionamiento de \verb`translateMatch`:
  \begin{itemize}
    \item[-]{\makebox[3.4cm]{$C \: x1 \: \ldots \: xn \: \rightarrow \: e$\hfill}}
    $\rightarrow$ Un patrón estándar con la constructora y $n>=0$ variables se traduce sin
    complicaciones. Se puede usar la misma estructura que en Haskell.
    \verb`C x1 ... xn -> translate e`.

    \item[-]\makebox[3.4cm]{$v \: \rightarrow \: e$\hfill} $\rightarrow$ Una simple variable
    en el patrón. Aquí surge la complicación y además un caso especial. Hay que tener en
    cuenta si la variable \verb`v` aparece en la expresión \verb`e`.
    \begin{itemize}
      \item[-] Si aparece, hay que volver a meter este valor en una mónada y crear un nuevo
      ámbito de variables con una lambda para sustituir el anterior nombre. Así, este caso
      se tiene que traducir como: \verb`v -> (\v -> translate e) (return v)`.
      \item[-] Si no aparece, para evitar que la inferencia de tipos de Haskell se encuentre
      con ambigüedades, hay que traducir este caso de la siguiente forma:
      \verb`_ -> translate e`.
    \end{itemize}
  \end{itemize}
\end{itemize}

Por último, para traducir una función hay que traducir su tipo y su expresión asociada.
Una función con argumentos se puede entender como una secuencia de expresiones lambda. Por
lo tanto, traducir \verb`f = e` consiste en traducir su tipo, \verb`f :: translateType f` y
su expresión, \verb`f = translate e`. Las funciones con más de un argumento se traducen como
una sucesión de expresiones lambda.

La traducción es mejorable, ya que recae muchísimo en el uso de la programación monádica
para reflejar el indeterminismo, pero se pueden llevar a cabo optimizaciones para mejorar
el rendimiento. El impacto en el rendimiento de la primera mejora puede ser cercana al 50\%,
como comentaremos más adelante en el apartado de conclusiones.

La primera tarea se puede realizar para evitar el uso excesivo de la programación monádica
es tener en cuenta una consecuencia lógica de las leyes del `call-time choice' para eliminar
por completo la compleja estructura de las funciones monádicas. Recordemos que actualmente
una función de mi lenguaje traducida a Haskell tiene el tipo
\verb`s (s a1 -> s (s a2 -> ... (s an)...))`. Este tipo es equivalente a
\verb`s a1 -> ... -> s an`. Es decir, eliminamos recursivamente del tipo de la función la
mónada que las envuelve. Esto es así ya que, según las leyes del `call-time choice', siendo
\verb`f` y \verb`g` funciones del mismo tipo, $\llbracket C[(f \: ? \: g) \: x]\rrbracket
\: = \: \llbracket C[f \: x]\rrbracket \: \bigcup \: \llbracket C[g \: x]\rrbracket$. Es
decir, claramente nos da igual elegir entre las dos funciones y juntar los resultados que
aplicar las dos funciones por separado y juntar los resultados. En la semántica
`run-time choice' también se cumple si consideramos que todas las apariciones de un
identificador son  independientes (que es básicamente el comportamiento de esta semántica).

La segunda optimización a realizar para mejorar el rendimiento es tener en cuenta en tiempo
de compilación si una subexpresión es determinista o indeterminista. La propiedad de ser
determinista es una propiedad semántica y por tanto típicamente indecidible, pero podrían
usarse aproximaciones sintácticas simples basadas en el hecho de que la fuente sintáctica
del indeterminismo es el operador \verb`(?)`. En el árbol de sintaxis abstracto se puede
anotar esta propiedad, de manera que mientras un cómputo sea determinista no se llevará a
cabo ninguna sobrecarga, usando funciones Haskell sin decorar, es decir, sin recurrir a la
programación monádica. Si alguna pieza es indeterminista, hay que traducir según las reglas
explicadas anteriormente. Esto permitiría que las piezas deterministas se ejecuten a la
máxima velocidad y las piezas no deterministas casi inevitablemente más lentas.

Estas tareas no se han llevado a cabo por la necesidad de acotar el desarrollo del lenguaje,
ya que no se persigue la eficiencia sino mostrar una posible solución simple a la combinación
de dos tipos de semántica y mostrar las ventajas que esto puede llegar a ofrecer.

\end{document}
