\documentclass[class=article, crop=false]{standalone}
\usepackage[utf8]{inputenc}
\usepackage[spanish]{babel}
\usepackage{minted}
\usepackage{xcolor}

\definecolor{bg}{rgb}{0.95,0.95,0.95}

\begin{document}

\section{Conclusions}
We have introduced and implemented a small functional language with indeterministic features
and a combination of `call-time choice' and `runtime choice' semantics, which was the core of
the project. The implementation of all phases has been done in Haskell, a purely functional
language of reference in this paradigm in the academic world. The translation makes intensive
use of monadic programming to represent indeterminism and the sharing of expressions, using
the primitive \verb`share` of Fischer's library. We also believe that translation is simple,
since each expression in Sharade corresponds to an expression in Haskell, following a very
basic recursive model and easy to understand.

Once again Haskell demonstrates his enormous potential and abstraction to describe in a clear
and concise way very complicated and not trivial computations. Thanks to his strong type
system it ensures great security about the result. You can even reason about the behavior of
the program just by looking at the types of functions that are involved. This is a great
advantage, but it is necessary to develop a certain fluency in order to be able to make such
deductions. At first the programmer may encounter cryptic errors and therefore have the
feeling that Haskell is a complicated language and does not help the programmer, but quite
the opposite. Even errors are providing great information if you know how to interpret.

It also needs to be mentioned that any unexplored Haskell feature for a programmer can be
something completely new, strange and without any solid concept on which to build to
understand the new aspect. This makes me consider that, being one of my favorite languages,
it is by far the most complicated language, even more so than the new C++ revisions. Haskell
is such a well-defined language that even types have types!

Finally, we believe that the combination of semantics in Sharade has remained with a clear
and consistent syntax, being able to express indeterministic algorithms with a quite
important semantic richness, reason why we believe that this combination or similar models
can be the base of future works.

\end{document}
