\documentclass[class=article, crop=false]{standalone}
\usepackage[utf8]{inputenc}
\usepackage[spanish]{babel}
\usepackage{minted}
\usepackage{xcolor}

\definecolor{bg}{rgb}{0.95,0.95,0.95}

\begin{document}

\section{Conclusiones y trabajo futuro}
Hemos presentado e implementado un pequeño lenguaje funcional con características
indeterministas y con combinación de las semánticas `call-time choice' y `run-time choice',
que era el núcleo del proyecto. La implementación de todas las fases se ha hecho en Haskell,
un lenguaje puramente funcional de referencia en este paradigma. La parte esencial de la
implementación viene dada por un proceso de traducción a Haskell de un programa Sharade; es
decir, lo que podemos considerar como un proceso de compilación a Haskell de Sharade. La
traducción hace un uso intensivo de programación monádica para representar el indeterminismo
y la compartición de expresiones, usando la primitiva \verb`share` de la biblioteca de
Fischer. También creemos que la traducción es sencilla, ya que cada expresión en Sharade se
corresponde con una expresión en Haskell, siguiendo un modelo recursivo muy básico y fácil de
entender.

Una vez más queda demostrado el enorme potencial y abstracción de Haskell para describir de
una manera clara y concisa cómputos muy complicados y a priori nada triviales. Gracias a su
sistema de tipos fuerte asegura gran seguridad acerca del resultado. Incluso se puede razonar
acerca del comportamiento del programa tan solo mirando los tipos de las funciones que están
involucradas. Esto supone una gran ventaja, pero es necesario desarrollar una cierta soltura
para poder realizar ese tipo de deducciones. Al principio el programador se puede encontrar
con errores crípticos y por lo tanto tener la sensación de que Haskell es un lenguaje
complicado y que no ayuda al programador, pero todo lo contrario. Incluso los errores están
aportando una gran información si se saben interpretar.

También es necesario citar que cualquier páramo inexplorado de Haskell para un programador
puede resultar algo completamente nuevo, extraño y sin ningún concepto sólido sobre el que
basarse para entender el nuevo aspecto. Esto me hace considerar que, siendo uno de mis
lenguajes favoritos, es el más complicado con diferencia, incluso más que las nuevas
revisiones de C++. ¡Haskell es un lenguaje tan bien definido que hasta los tipos tienen
tipos!

Por último, creemos que la combinación de semánticas en Sharade ha quedado con una sintaxis
clara y consistente, pudiendo expresar algoritmos indeterministas con una riqueza semántica
bastante importante, por lo que creemos que esta combinación o modelos similares puedan ser
la base de futuros trabajos sobre programación con funciones no deterministas.

Aunque el trabajo aquí desarrollado satisface los objetivos propuestos, dista mucho de agotar
todas las posibilidades del tema abordado. De hecho, hay mucho espacio para nuevos
desarrollos. Tenemos claro que algunos aspectos admiten mejoras, de las cuales daremos pistas
sobre cómo se pueden llevar a cabo e incorporar al trabajo existente, como examinamos a
continuación.

\begin{itemize}
  \item[-] En cuanto a la sintaxis, se puede añadir mucho azúcar sintáctico para las listas,
  listas intensionales, múltiples definiciones ecuacionales para sustituir las expresiones
  \verb`case`, etc.

  \item[-] Podría ser interesante dotar al lenguaje un sistema de módulos como en Haskell,
  añadir funciones predefinidas y aumentar las características de las listas.

  \item[-] Para mejorar el rendimiento se pueden poner en práctica las dos propuestas
  realizadas, que como se ha mostrado en un ejemplo, puede mejorar el tiempo de ejecución en
  un 50\%. La primera propuesta está parcialmente desarrollada en la rama \verb`optv1` del
  repositorio.

  \item[-] También se puede realizar la derivación automática de clases, con alguna de las
  herramientas propuestas o actualizando la implementación de Fischer. Esto luego se tendría
  que incorporar al proceso de traducción. Una vez hecho esto, podrían incorporarse clases de
  tipos a Sharade, pudiendo unificar los operadores de suma, resta... de los tipos
  \verb`Integer` y \verb`Double`, modificando así el sistema de inferencia de
  tipos\cite{smith1994principal}.

  \item[-] Se puede hacer más evidente la incorporación de nuevas formas de recolectar los
  resultados. Actualmente está escondido y requiere un conocimiento básico del funcionamiento
  de la biblioteca de Fischer para implementarlos.

  \item[-] Es necesario añadir más funcionalidades al compilador. Actualmente no permite
  nombrar el fichero de salida, no tiene opciones para crearlo como un módulo con nombre
  propio y podría ser interesante incorporar a la traducción funciones auxiliares para no
  tener que tratar directamente con el código traducido.
\end{itemize}

\subsection{Rendimiento}\label{sec:rendimiento}

El rendimiento de Sharade es mejorable. El ejemplo de `permutation sort' de
Fischer\cite{fischer2011purely} tarda 16 segundos en ordenar una lista de 20 elementos, un
rendimiento muy parecido al que se obtiene en Curry con MCC\cite{lux2003munster}. Sin embargo
Sharade tarda en ordenar la misma lista 43 segundos.

Esto es debido al intensivo uso de la programación monádica. En la implementación de Fischer,
las funciones que implementa son deterministas, en Sharade todas deben ser no deterministas.
Esto produce una sobrecarga bastante importante de programación monádica, lo que lleva a una
pérdida del rendimiento de cerca del 50\%. Fischer en su biblioteca también hace un uso
intensivo de la programación monádica, pero es capaz de llevar a cabo muchas optimizaciones
en el código para reducir este uso, tarea que no hemos realizado en nuestra implementación de
la traducción.

Aplicando a mano la primera optimización propuesta en el apartado de `Traducción a Haskell',
el ejemplo de `permutationsort' observa una mejora de rendimiento muy notable, más del 50\%,
tardando 18 segundos. La segunda optimización no se puede aplicar en este ejemplo ya que es
intrínsecamente indeterminista.

\begin{minted}[bgcolor=bg]{haskell}
import Sharade.Prelude

insert :: Sharing s => s a -> s (List s a) -> s (List s a)
insert e ls = mPlus (cons e ls) (ls >>= (\pec -> case pec of
  Cons x xs -> cons x (insert e xs)
  _ -> mzero))

perm :: Sharing s => s (List s a) -> s (List s a)
perm ls = ls >>= (\pec -> case pec of
  Nil -> nil
  Cons x xs -> insert x (perm xs))

isSorted :: Sharing s => s (List s Integer) -> s Bool
isSorted ls = ls >>= (\pec -> case pec of
  Nil -> true
  Cons x xs -> xs >>= (\pec -> case pec of
    Nil -> true
    Cons y ys -> mAnd (mLeq x y) (isSorted cons y ys)))

sort :: Sharing s => s (List s Integer) -> s (List s Integer)
sort ls = share (perm ls) >>= (\sls -> (isSorted sls) >>= (\pec ->
  case pec of
    True -> sls
    _ -> mzero))
\end{minted}

Como se explica en la traducción, cada aplicación funcional se debe hacer con el operador
infijo \verb`<#>`, ya que la función esta escondida en una mónada y también el valor.
Quitando toda esa sobrecarga y devolviendo a las funciones su aspecto habitual, queda un
código bastante parecido al de Fischer. Como ya hemos dicho antes, tarda en ordenar una lista
de 20 elementos 18 segundos, una marca muy cercana a la que consigue Fischer en su
implementación. Este resultado me resulta bastante satisfactorio, ya que es una optimización
relativamente fácil de llevar a cabo en un futuro. De hecho, hay una implementación parcial
de esta idea en la rama \verb`optv1` del repositorio de GitHub. Es parcial porque no realiza
compartición de valores funcionales. Salvando eso, el resto de características del lenguaje
funcionan correctamente, incluido el orden superior y, por supuesto, la compartición de tipos
primitivos.

\subsection{Recolección de resultados}

La forma de recolectar los resultados parece ser bastante importante en el mundo de la
programación indeterminista. Por ejemplo, el compilador KiCS2 de Curry ofrece varias formas
de hacer la búsqueda en el espacio de resultados\cite{hanus2012search}.

Sharade proporciona dos formas de recolectar los resultados, explicadas anteriormente,
que son las implementaciones de Fischer. Pero también se da al usuario la posibilidad de
implementar sus propias fórmulas de búsqueda, dando la posibilidad de descartar resultados
para aumentar el rendimiento, como una búsqueda de ramificación y poda. También se pueden
implementar búsquedas más simples. como una búsqueda en anchura, en profundidad iterativa o
una búsqueda paralela multihilo. Por último quería destacar que no hay ningún motivo técnico
en la implementación que pueda hacer que una búsqueda que no cumpla con las leyes de Fischer
provoque un mal funcionamiento de la biblioteca y, por extensión, en nuestra implementación.

Fischer tiene un trabajo\cite{fischer2009reinventing} sobre cómo modificar el sistema de
backtracking en programación monádica en Haskell. En concreto, muestra cómo implementar la
búsqueda en anchura y la búsqueda en profundidad por niveles, además de reinventar la
búsqueda en profundidad que hace Haskell. Estas ideas se pueden usar para implementar tu
propia mónada que realice la búsqueda deseada e incorporarla a Sharade.

\subsection{Tipos de usuario}

Sharade no soporta la definición de tipos arbitrarios. Esto es debido a que esos tipos serían
indeterministas, por lo que tienen que estar sujetos a las reglas del Sharing de Fischer y
pertenecer a una serie de clases de tipos. Fischer tiene implementado una derivación
automática para tipos arbitrarios a sus clases de tipos, pero no funciona con las versiones
actuales de Haskell. Actualmente se puede conseguir con múltiples herramientas como
\verb`Template-Haskell`\cite{sheard2002template} o con
\verb`DrIFT`\cite{DrIFT} o con \verb`derive`\cite{derive}. Esta última es la que usa Fischer
para implementar la derivación automática de las clases de tipos. Abordar esta cuestión
estaba fuera de los objetivos, debido a la necesidad de acotar el trabajo tanto en tiempo
como en características. Por lo tanto, Sharade no soporta tipos de usuario, dejándolo como un
trabajo futuro.

\end{document}
