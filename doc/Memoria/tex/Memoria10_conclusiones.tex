\documentclass[class=article, crop=false]{standalone}
\usepackage[utf8]{inputenc}
\usepackage[spanish]{babel}
\usepackage{minted}
\usepackage{xcolor}

\definecolor{bg}{rgb}{0.95,0.95,0.95}

\begin{document}

\section{Conclusiones}
Hemos presentado e implementado un pequeño lenguaje funcional con características
indeterministas y con combinación de las semánticas `call-time choice' y `runtime choice',
que era el núcleo del proyecto. La implementación de todas las fases se ha hecho en Haskell,
un lenguaje puramente funcional de referencia en este paradigma en el mundo académico. La
traducción hace un uso intensivo de programación monádica para representar el indeterminismo
y la compartición de expresiones, usando la primitiva \verb`share` de la biblioteca de
Fischer. También creemos que la traducción es sencilla, ya que cada expresión en Sharade se
corresponde con una expresión en Haskell, siguiendo un modelo recursivo muy básico y fácil de
entender.

Una vez más queda demostrado el enorme potencial y abstracción de Haskell para describir de
una manera clara y concisa cómputos muy complicados y a priori nada triviales. Gracias a su
sistema de tipos fuerte asegura gran seguridad acerca del resultado. Incluso se puede razonar
acerca del comportamiento del programa tan solo mirando los tipos de las funciones que están
involucradas. Esto supone una gran ventaja, pero es necesario desarrollar una cierta soltura
para poder realizar ese tipo de deducciones. Al principio el programador se puede encontrar
con errores crípticos y por lo tanto tener la sensación de que Haskell es un lenguaje
complicado y que no ayuda al programador, pero todo lo contrario. Incluso los errores están
aportando una gran información si se saben interpretar.

También es necesario citar que cualquier páramo inexplorado de Haskell para un programador
puede resultar algo completamente nuevo, extraño y sin ningún concepto sólido sobre el que
basarse para entender el nuevo aspecto. Esto me hace considerar que, siendo uno de mis
lenguajes favoritos, es el más complicado con diferencia, incluso más que las nuevas
revisiones de C++. ¡Haskell es un lenguaje tan bien definido que hasta los tipos tienen
tipos!

Por último, creemos que la combinación de semánticas en Sharade ha quedado con una sintaxis
clara y consistente, pudiendo expresar algoritmos indeterministas con una riqueza semántica
bastante importante, por lo que creemos que esta combinación o modelos similares puedan ser
la base de futuros trabajos.

\end{document}
