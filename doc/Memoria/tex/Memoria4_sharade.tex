\documentclass[class=article, crop=false]{standalone}
\usepackage[utf8]{inputenc}
\usepackage[spanish]{babel}
\usepackage{minted}
\usepackage{xcolor}

\definecolor{bg}{rgb}{0.95,0.95,0.95}

\begin{document}

\section{Sharade}
\subsection{Introducción}

El resultado de este trabajo de fin de grado es un lenguaje funcional con características
habituales en el paradigma de la programación funcional como el orden superior, evaluación
perezosa, ajuste de patrones, expresiones lambda, no determinismo, semánticas de compartición
(call-time choice) y no compartición (runtime choice). Posee la inferencia de tipos de
Hindley-Milner pero no se pueden especificar los tipos en la declaración de las funciones. No
tiene clases de tipos ni permite crear tipos de usuario por cuestiones explicadas más
adelante. La sintaxis está muy influenciada por Haskell, tiene aplicación parcial (excepto
para operadores infijos) y notación currificada. El lenguaje carece de algunos azúcares
sintácticos habituales, por ejemplo, las listas se tienen que escribir a base de las
constructoras \verb`Cons` y \verb`Nil`, no existe ajuste de patrones en los lados izquierdos
de la definición de funciones, estando delegados a una  expresión \verb`case` en las
expresiones de los lados derechos ni tampoco existen listas intensionales. No posee ninguna
biblioteca estándar ni preludio, todas las funciones básicas deben ser programadas por el
usuario. En cuanto a los tipos primitivos disponibles (y únicos) están los enteros, los
reales, los caracteres, las listas y las parejas. Todo esto es debido a que se trata de un
trabajo de fin de grado donde el objetivo era mostrar de una forma simple y clara la
combinación de los dos tipos de semánticas. Esta combinación puede ser muy útil, ya que hay
situaciones prácticas donde ninguna de las dos semánticas son apropiadas pero sí una
combinación de ellas\cite{lopez2009flexible}, como veremos en los ejemplos más
adelante. La combinación también permite describir algoritmos con mayor flexibilidad y
otorgando más control al programador. Tal como hemos decidido definir la sintaxis, la
combinación resulta muy sencilla y limpia, como veremos en el siguiente apartado.

Existen lenguajes más avanzados con más características y con mejor rendimiento que el mío,
como pueden ser Curry y Toy. Sharade no persigue el rendimiento (aunque con algunas
optimizaciones puede llegar a tener un rendimiento similar) ya que su objetivo es incorporar
a la literatura de la comunidad lógico-funcional algo nuevo, la combinación de dos semánticas
indeterministas, el `call-time choice' y `runtime choice'. Esto se consigue, desde el punto
de vista de la estructura del lenguaje, asumiendo por defecto un comportamiento de semántica
de `runtime choice' y a la incorporación de una primitiva que veremos en el siguiente
apartado mediante ejemplos.

En cuanto a la implementación, que comentaremos durante capítulos sucesivos, la basaremos en
una traducción de Sharade a Haskell que hace un uso intensivo de la programación monádica
para expresar el indeterminismo.

\subsection{Un vistazo a Sharade}

En los siguientes ejemplos voy a mostrar la sintaxis y características de Sharade y cómo se
pueden combinar entre ellas. Además, también voy a mostrar dos programas algo más elaborados,
como la implementación de un autómata finito no determinista y un tipo de ordenación de
listas algo peculiar, la ordenación por permutación. Este último ejemplo se usa bastante en
la literatura para comparar el rendimiento de los lenguajes no deterministas, ya que usa
intensivamente el indeterminismo y tiene gran carga computacional. Se basa en escoger la
permutación ordenada de una lista.

\subsubsection{No determinismo}

Un ejemplo simple donde aflora el no-determinismo podría ser el siguiente:

\begin{minted}[bgcolor=bg]{haskell}
number = 1 ? 2 ? 3 ? 4 ;

duplicate x = x + x ;
\end{minted}

El operador infijo `\verb`?`' es el origen del indeterminismo en Sharade. Esta definición
significa que \verb`number` se puede reducir a uno de esos cuatro valores. La función
\verb`duplicate` tiene un argumento, que potencialmente puede ser no determinista. Es decir,
puede operar con un valor determinista (por ejemplo, un simple \verb`2`) o un valor no
determinista como \verb`number`. Tradicionalmente hay dos formas de entender este programa.
Si entiende con la semántica de `call-time choice', \verb`duplicate number` solo tendrá 4
resultados. Sin embargo, con la semántica `runtime choice' se obtienen 16 resultados,
provenientes de todas las combinaciones posibles de los valores de \verb`number`.

Sharade por defecto opera con la semántica de `runtime choice', por lo tanto la expresión
\verb`duplicate number` tiene 16 resultados, de los cuales 12 se repiten por la
conmutatividad de la suma. En el siguiente ejemplo vamos a ver como podemos operar con la
otra semántica.

\subsubsection{Compartición}

Para usar el otro tipo de semántica hay que introducir la primitiva `\verb`choose`'. Es una
primitiva parecida a una expresión `\verb`let`'. Crea un nuevo ámbito con una nueva variable
que cumplirá las leyes del `call-time choice'\cite{lopez2007simple}. Es decir, todas sus
apariciones en una expresión se evaluarán al mismo valor.

\begin{minted}[bgcolor=bg]{haskell}
number = 1 ? 2 ? 3 ? 4 ;

duplicate x = choose a = x in a + a ;
\end{minted}

En el ejemplo se está ligando la variable \verb`a` a uno de los posibles valores de \verb`x`.
De esta forma, la expresión \verb`duplicate number` solo tiene 4 posibles valores, igual que
en Curry.

\subsubsection{Combinación de semánticas}

En el siguiente ejemplo se puede ver la combinación de los dos tipos de semántica, un salto
que ha dado Sharade:

\begin{minted}[bgcolor=bg]{haskell}
number = 1 ? 2 ? 3 ? 4 ;
coin = 0 ? 1 ;

f x y = choose a = x in a + a + y + y ;
\end{minted}

Si se evalúa \verb`duplicate number coin` existen 16 combinaciones. Esto es debido a que la
variable `\verb`y`' está siguiendo las reglas del `runtime choice' y la variable `\verb`a`'
las del `call-time choice'.

Se puede pensar que una variable sujeta a la semántica de `call-time choice' es una variable
normal de cualquier lenguaje, sus apariciones comparten el valor a lo largo de la expresión.

Veamos otro ejemplo práctico donde además se introducen listas infinitas e infinitos
resultados indeterministas:

\begin{minted}[bgcolor=bg]{haskell}
letter = 'a' ? 'b' ? 'c' ;

word = Nil ? Cons letter word ;

concatenate xs ys = case xs of
  Nil -> ys ;
  Cons x xs -> Cons x (concatenate xs ys) ;;

reverse' xs ys = case xs of
  Nil -> ys ;
  Cons x xs -> reverse' xs (Cons x ys) ;;

mreverse xs = reverse' xs Nil ;

palindrome = choose w = word in concatenate w (mreverse w) ;
\end{minted}

Las únicas piezas no deterministas en este programa son las definiciones de \verb`letter`,
\verb`word` y \verb`palindrome`. Una letra puede tener cuatro posibles valores y una palabra
puede ser la cadena vacía (\verb`Nil`) o una letra seguida de una palabra. Así, una letra
tiene el tipo \verb`Char` y una palabra el tipo \verb`[Char]`. Un palíndromo es una palabra
concatenada con su inversa. Por simplicidad en el ejemplo, no he considerado los palíndromos
de longitud impar. Las funciones auxiliares son deterministas y no difieren en nada de como
se programarían en un lenguaje funcional tradicional. En este ejemplo se puede ver la
combinación de las dos semánticas, `runtime choice' en \verb`word` y `call-time
choice' en \verb`palindrome`. Completemos el ejemplo:

\begin{minted}[bgcolor=bg]{haskell}
mlength xs = case xs of
  Nil -> 0 ;
  Cons x xs -> 1 + mlength xs ;;

e1 = choose p = palindrome in case mlength p == 4 of
  True -> p ;;
\end{minted}

La función \verb`mlength` es clásica. Si evaluamos \verb`e1` se calcularán todos los
palíndromos de longitud cuatro pero desgraciadamente el cómputo nunca terminará. Esto es
debido a una limitación de Haskell y se debe a la misma razón por la cual la expresión
\verb`filter (\a -> length a == 4) (iterate (1:) [])` no termina.

\subsubsection{Orden superior}

Sharade también tiene características de orden superior, indispensables en un lenguaje
funcional. El orden superior se puede combinar con el no determinismo, con el operador
`\verb`?`' y con la primitiva \verb`choose`.

\begin{minted}[bgcolor=bg]{haskell}
f a b = a + b ;
g a b = a * b ;

h z a b = z a b ;
\end{minted}

En el ejemplo se crea una función \verb`f` idéntica a la función suma, una función \verb`g`
idéntica a la multiplicación y una función \verb`h`, con tres argumentos. El primero es una
función que se aplica al segundo y al tercero. De esta forma, se puede pasar como argumento
una función no determinista e incluso aprovechar las expresiones lambda. También es posible
combinar esta característica con la primitiva \verb`choose`:

\begin{minted}[bgcolor=bg]{haskell}
e1 = h (f ? g) 1 2 ;
e2 = choose f' = f ? g in f' (f' 1 2) (f' 1 2) ;
\end{minted}

La definición de \verb`e1` es muy simple y tendrá como resultado dos valores, \verb`3` y
\verb`2`. Con \verb`e2` se puede ver de nuevo en acción la primitiva \verb`choose` ligando
\verb`f'`. De esta manera, los dos únicos valores posibles son \verb`(1 + 2) + (1 + 2) = 6` y
\verb`(1 * 2) * (1 * 2) = 4`.

\subsubsection{Ajuste de patrones}

Haskell permite usar ajuste de patrones en la definición de las funciones, pero no es más que
azúcar sintáctico para expresiones \verb`case` muy tediosas. Por simplicidad, Sharade no
cuenta con este azúcar sintáctico.

\begin{minted}[bgcolor=bg]{haskell}
f a = case a of
  0 -> 1 ;
  1 -> 2 ;
  v -> v + v ;;

fact n = case n of
  1 -> 1 ;
  n -> n * fact (n - 1) ;;
\end{minted}

El primer ejemplo es el interesante, el segundo es una implementación simple del cálculo del
factorial, no emerge ninguna característica del no determinismo y menos de la compartición.
Se deja ver que Sharade también soporta recursión, un añadido a la inferencia de tipos de
Hindley-Milner. En el primer ejemplo, igual que en la primitiva \verb`choose`, el último
patrón crea un nuevo ámbito con la variable \verb`v` disponible. En este caso también aplican
las leyes del `call-time choice', pero no con buenas propiedades, como la demanda tardía y la
ley `ignore', como veremos más adelante en la implementación. Para entenderlo rápidamente,
se pierde la evaluación perezosa cuando es obligatorio evaluar \verb`a`.

\subsubsection{Autómata finito no determinista}

Un primer ejemplo de la potencia que permite la programación no determinista podría ser la
implementación de un autómata finito no determinista de una forma muy sencilla.

\begin{minted}[bgcolor=bg]{haskell}
-- f :: State -> Char -> State
f s a = case s of
  0 -> case a of
    'a' -> 0 ;
    'b' -> 0 ? 1 ;;
  1 -> case a of
    'a' -> 2 ;;;

-- accept' :: (State -> Char -> State) -> State -> [Char] -> Bool
accept' f s as = case as of
  Nil       -> s == 2 ;
  Cons a as -> accept' f (f s a) as ;;

-- accept :: (State -> Char -> State) -> [Char] -> Bool
accept f as = accept' f 0 as ;
\end{minted}

Aquí se ven en juego todas las características previamente mencionadas. El ajuste de
patrones, orden superior y no determinismo. No se ve ningún ejemplo de semántica `call-time
choice', ya que en este ejemplo no tiene cabida, se quiere usar las características que
proporciona la semántica `runtime choice'.

Este AFN reconoce las cadenas de caracteres terminadas en `ba'. La función \verb`f` es la
función de transición. La función de transición en un AFN es no determinista y en este
ejemplo esto se puede apreciar en el primer estado cuando se lee una `b'. En un lenguaje
tradicional se tendría que trabajar con un conjunto de valores, pero en este paradigma eso se
da por supuesto, permitiendo una potencia descriptiva del algoritmo bastante atractiva. La
función \verb`accept'` aplica recursivamente la función de transición, aceptando únicamente
si se ha consumido toda la entrada y el autómata se encuentra en el estado 2.

\subsubsection{Ordenación por permutación}

Este ejemplo es el que usa Fischer para mostrar la biblioteca en la que me he basado para
construir Sharade. Este tipo de ordenación es simplemente para un ejemplo y no tiene
ningún uso práctico, ya que este algoritmo de ordenación es extremadamente ineficiente.
Consiste en calcular todas las permutaciones de una lista y comprobar si alguna de ellas está
ordenada. Con la evaluación perezosa quizás no es necesario calcular toda la permutación
completa, ya que solo con mirar los dos primeros elementos de la lista puedes descartar todo
el subárbol de permutaciones que se podrían generar. Fischer lo utiliza para mostrar que
funciona la evaluación perezosa y por tanto se pueden descartar permutaciones sin necesidad
de calcularlas enteras. Lo he querido poner como ejemplo para mostrar de nuevo la semántica
`call-time choice' de Sharade y posteriormente para analizar la pérdida de rendimiento de
Sharade. Como el ejemplo puede resultar complejo la primera vez, he decidido separarlo por
trozos para explicarlo mejor.

\begin{minted}[bgcolor=bg]{haskell}
insert e ls = (Cons e ls) ? case ls of
  Cons x xs -> Cons x (insert e xs) ;;
\end{minted}

Esta función `insertar' recibe un elemento y una lista. Inserta el elemento en algún lugar de
la lista. Por lo tanto, se puede poner en la cabeza de la lista o insertarlo en algún lugar
de la cola, llamando recursivamente.

\begin{minted}[bgcolor=bg]{haskell}
perm ls = case ls of
  Nil -> Nil ;
  Cons x xs -> insert x (perm xs) ;;
\end{minted}

Esta función calcula todas las permutaciones de una lista dada, apoyándose en la función
\verb`insert`y llamándose recursivamente. Se calcula por tanto insertando la cabeza en todas
las posiciones posibles de la permutación de la cola.

\begin{minted}[bgcolor=bg]{haskell}
isSorted ls = case ls of
  Nil -> True ;
  Cons x xs -> case xs of
    Nil -> True ;
    Cons y ys -> x <= y && isSorted (Cons y ys) ;;;

sort ls = choose sls = perm ls in case isSorted sls of
  True -> sls ;;
\end{minted}

La función \verb`isSorted` es compleja de seguir debido a todo el ajuste de patrones que hay
que hacer pero no difiere en nada de como se haría en un lenguaje funcional tradicional.
Simplemente calcula si una lista dada está ordenada. La segunda función, \verb`sort`, calcula
una permutación y comprueba si está ordenada, en tal caso la devuelve. Para ello hay que usar
la primitiva \verb`choose`, ya que \verb`sls` aparece dos veces en la expresión y debe
compartir el valor porque tiene que pasar el filtro de si está ordenada.

En resumen, el ejemplo calcula todas las permutaciones y devuelve la que está ordenada.
También se aprovecha de las características como la evaluación perezosa que Fischer consigue
traer al no determinismo en Haskell.

\subsubsection{Evaluación perezosa}

Para terminar, me gustaría mostrar que Sharade evalúa perezosamente y por tanto puede tratar
con estructuras de datos infinitas, resultados que podrían dar error o cómputos infinitos.

\begin{minted}[bgcolor=bg]{haskell}
f x y = y ;
e0 = f (1/0 ? 2/0) 0 ;
\end{minted}

Si se evalúa \verb`e0` da como resultado \verb`0`, el segundo argumento, no da ningún error
por dividir por cero, tal como lo haría Haskell. Esto es debido a que la función no es
estricta en el primer argumento. Esta propiedad es muy importante, Fischer la llama `Ignore'.
Explicaré en el aparatado `Biblioteca explicit-sharing' cómo se consigue esto en la
implementación. Veamos el siguiente ejemplo:

\begin{minted}[bgcolor=bg]{haskell}
mrepeat x = Cons x (mrepeat x) ;
mtake n xs = case n of
  0 -> Nil ;
  n -> case xs of
    Nil -> Nil ;
    Cons y ys -> Cons y (mtake (n - 1) ys) ;;;

coin = 0 ? 1 ;
e1 = mtake 3 (mrepeat coin) ;
e2 = choose c = coin in mtake 3 (mrepeat c) ;
e3 = let c = coin in mtake 3 (mrepeat c) ;
\end{minted}

Las funciones \verb`mrepeat` y \verb`mtake` son deterministas. La primera crea una lista
infinita con el primer argumento en cada posición y la segunda recoge los primeros \verb`n`
elementos de una lista. La primera y la tercera expresión, \verb`e1` y \verb`e3`, son
idénticas. El cómputo termina y da como resultado un conjunto de ocho resultados, ya que las
dos expresiones son equivalentes a \verb`[coin, coin, coin]` y como cada coin es
independiente se generan los ocho resultados.

La segunda expresión, \verb`e2`, también termina pero da como resultado dos valores. Esto es
normal porque, como ya veremos, la expresión es equivalente a
\verb`choose c = coin in [c, c, c]`, por lo que el valor de todas las \verb`c` se comparte.

En resumen, Sharade consigue mantener las buenas propiedades que Fischer consigue con su
implementación en Haskell, en este ejemplo concreto, con la evaluación perezosa. Otra
propiedad que mantiene, esta vez gracias a Haskell, es la compartición de los argumentos de
una función, de manera que si uno de ellos tarda mucho en computarse, se puede reutilizar ese
valor en todas las apariciones de dicho argumento.

\subsection{Recolección de resultados}

La implementación de Fischer provee dos formas de recolectar los resultados en Haskell. Sobre
cómo usar el código generado hablaremos más adelante. Estas dos formas son con la función
\verb`results` y otra con \verb`unsafeResults`. Para entender a qué se refiere Fischer con
`unsafe' hay que entender las siguientes leyes:

\begin{itemize}
	\item[-] Idempotencia: \verb`a ? a = a`
	\item[-] Conmutativa: \verb`(a ? b) ? (c ? d) = (a ? c) ? (b ? d)`
\end{itemize}

Cualquier implementación que recolecte los resultados que no cumpla esas leyes es considerada
`insegura'. Con estas leyes, una recolección de resultados por anchura o por profundidad
sería considerada insegura, ya que no cumpliría la segunda ley. Además, también se tendría
que omitir resultados, ya que no se pueden repetir.

La primera implementación de Fischer se basa en un conjunto ordenado de Haskell 
(\verb`Set a`). Los resultados no se almacenan de acuerdo con la recolección de ellos. Otro
detalle es que se tienen que recolectar todos los resultados, por lo que se carece de
evaluación perezosa y en caso de que haya infinitos resultados, la ejecución nunca terminará.
De hecho, ni si quiera se podrá obtener el primero de ellos.

La segunda implementación, la insegura, usa las listas estándar de Haskell. Al contrario que
la anterior, posee evaluación perezosa, puede tratar con resultados infinitos y es
equivalente a una búsqueda en profundidad por la izquierda.

\subsection{Sintaxis abstracta}\label{sec:sintaxis_abstracta}

Los ejemplos anteriores han permitido introducir buena parte de los elementos sintácticos
de Sharade, así como su sentido dentro de los objetivos de este trabajo. En este apartado
mostramos una visión más completa y unificada de las diferentes construcciones sintácticas
del lenguaje para posterior referencia, dejando los detalles más concretos de la sintaxis al
apartado \ref{sec:sintaxis_concreta}, al describir las distintas componentes de la
implementación.

Un programa de Sharade consiste en una serie de definiciones de funciones de la forma
\verb`f x1 ... xn = e ;`, con $n >= 0$, siendo \verb`f, x1 ... xn` identificadores que sirven
tanto para variables como funciones (permitiendo orden superior) y \verb`e` es una expresión.
Cada nombre de función \verb`f` dispone de una única ecuación definicional, por lo tanto el
ajuste de patrones y el indeterminismo se concentran en la expresión \verb`e`. Una expresión
\verb`e` adopta una de las siguientes estructuras:

\begin{itemize}
  \item[-]{\makebox[1.2cm]{$l$\hfill}} $\Rightarrow$ Un literal, que puede ser True, False y
  un número entero o real.
  
  \item[-]{\makebox[1.2cm]{$x$\hfill}} $\Rightarrow$ Una variable, siempre empiezan con
  minúscula.
  
  \item[-]{\makebox[1.2cm]{$C$\hfill}} $\Rightarrow$ Una constructora de datos, siempre
  empiezan con mayúscula.
  
  \item[-]{\makebox[1.2cm]{$e1 \: e2$\hfill}} $\Rightarrow$ Una aplicación de funciones.
  \item[-]{\makebox[1.2cm]{$\backslash x \rightarrow e$\hfill}} $\Rightarrow$ Una función
  lambda.
  
  \item[-]{\makebox[1.2cm]{$e1 \:? \: e2$\hfill}}  $\Rightarrow$ Una elección de dos valores,
  origen del indeterminismo.
  
  \item[-]{\makebox[1.2cm]{$(e1, \: e2)$\hfill}} $\Rightarrow$ Parejas polimórficas.
  \item[-]{\makebox[3.4cm]{$choose \: x = e1 \: in \: e2$\hfill}} $\Rightarrow$ Compartición
  de la variable.
  
  \item[-]{\makebox[3.4cm]{$let \: x = e1 \: in \: e2$\hfill}} $\Rightarrow$ Creación de una
  ligadura.
  
  \item[-]{\makebox[3.4cm]{$e1 \: (infixOp) \: e2$\hfill}} $\Rightarrow$ Una expresión con un
  operador infijo. Los únicos operadores infijos que existen son $(+)$, $(-)$, $(*)$, $(/)$,
  $(\&\&)$ y $(||)$.

  \item[-]$case \: e \: of \: t1 \rightarrow e1 ; \ldots tn \rightarrow en ;$ $\Rightarrow$
  Donde \verb`t` es un patrón, que tiene la estructura \verb`C x1 ... xn` o \verb`(x1, x2)`,
  siendo C una constructora de datos y el segundo patrón es el de las parejas. Las únicas
  constructoras de datos que existen son los números, True, False, Cons, Nil y Pair.
\end{itemize}

Para las aplicaciones asumimos las convenciones habituales de la notación currificada, en la
que `$e1 \: e2 \ldots en$' es azúcar sintáctico para `$(\ldots(e1 \: e2)\ldots) \: en$'. Las
únicas funciones que existen en el lenguaje son $dAdd$, $dSub$, $dMul$, $dDiv$, ..., $dNeq$
para el tipo Double. Los operadores infijos son únicamente para los números enteros y
booleanos. Esto es así debido a que el lenguaje no tiene clases de tipos. Para la función
\verb`?` asumimos el tipo \verb`(?) :: a -> a -> a`.

Pedimos a los programas que estén bien tipados con relación al sistema de tipos
Hindley-Milner. Como tipos primitivos consideramos \verb`Bool`, \verb`Char`, \verb`Integer`,
\verb`Double`, las parejas polimórficas (\verb`(a, b))` y las listas (\verb`[a]`). Los tipos
no se pueden especificar en la definición de las funciones.

\subsection{Instalación y uso}

Instalar la biblioteca Sharade y el compilador es relativamente sencillo. Para empezar, se
necesita una versión especifica de una biblioteca que no existe en el repositorio de paquetes
de Haskell, Hackage. Esta biblioteca es \verb`explicit-sharing`. Su autor original, como ya
he dicho en múltiples ocasiones, es Fischer, pero he tenido que hacer unos retoques para
hacerla compatible con las nuevas versiones de Haskell. Para instalar esta biblioteca, se
puede usar la siguiente ristra de comandos:

\begin{minted}[bgcolor=bg]{bash}
# Clonar el repositorio
git clone git@github.com:ManuelVs/explicit-sharing.git
# Actualizar la lista de paquetes de cabal
cabal update
# Generar la biblioteca
cabal sdist
# Instalar la biblioteca
cabal install dist/explicit-sharing*.tar.gz
# Limpiar los resultados intermedios (opcional)
cabal clean
\end{minted}

Ahora hay que hacer algo parecido para instalar mi biblioteca y compilador:

\begin{minted}[bgcolor=bg]{bash}
# Clonar el repositorio
git clone git@github.com:ManuelVs/Sharade.git
# Actualizar la lista de paquetes de cabal
cabal update
# Generar la biblioteca
cabal sdist
# Instalar la biblioteca
cabal install dist/Sharade*.tar.gz
# Limpiar los resultados intermedios (opcional)
cabal clean
\end{minted}

En la instalación \verb`cabal` debe haber puesto el compilador de Sharade en alguna
localización accesible para la línea de comandos, añadiéndolo al \verb`PATH` o similar. En
caso de que no lo haya hecho, se puede hacer a mano. En un sistema Linux debería poder
encontrarse en \verb`/home/username/.cabal/bin`. En un sistema Windows puede encontrarse en
\verb`C:\Users\username\.cabal\bin`.

Para compilar un fichero tan solo hay que pasar como argumento la localización de dicho
fichero.

\begin{minted}[bgcolor=bg]{bash}
$ cat example.sa
mRepeat a = a ? mRepeat (a + 1) ;
$ Sharade example.sa
Done.
\end{minted}

\subsection{Uso en Haskell}

Para usar la traducción hay que conocer un repertorio de funciones básicas:

\begin{itemize}
  \item[-] \verb`results`: Recolecta los resultados en un conjunto ordenado de Haskell. Tiene
  el tipo \verb`results :: Sharing s => s a -> Set a`. Esta función asegura las leyes de
  Fischer, explicadas anteriormente.
  
  \item[-] \verb`unsafeResults`: Recolecta los resultados en una lista estándar de Haskell.
  Tiene el tipo \verb`unsafeResults :: Sharing s => s a -> [a]`.
  
  \item[-] Operador infijo \verb`<#>`: Necesario para hacer cualquier aplicación funcional
  con funciones traducidas de Sharade. Su existencia es explicada en un punto posterior de
  este documento. Por ejemplo, \verb`mRepeat <#> (return 1)`.
  
  \item[-] \verb`convert`: Convierte tipos primitivos de Haskell a Sharade y viceversa,
  incluyendo listas y parejas. El tipo es complejo e incorpora nuevas clases de tipos cuya
  explicación distraería el objetivo de este apartado. Para transformar los tipos de Haskell
  a los tipos de Sharade hay que usarla de la manera convencional, por ejemplo,
  \verb`convert [1,2,3::Integer]`. Es necesario especificar los tipos, ya que Haskell en
  ocasiones no es capaz de inferirlos (por estas clases de tipos). Para transformar los tipos
  en el otro sentido hay que usar el operador \verb`>>=`. Por ejemplo, \\
  \verb`sort <#> (convert [1..20::Integer]) >>= convert`.
\end{itemize}

De esta manera, para usar la función compilada en el ejemplo de \verb`mRepeat` se puede usar
la siguiente expresión Haskell:

\begin{minted}[bgcolor=bg]{haskell}
e0 = unsafeResults $ mRepeat <#> (return 1) :: [Integer]
\end{minted}

Para evaluar la función \verb`sort` del ejemplo de ordenación por permutación se puede usar:

\begin{minted}[bgcolor=bg]{haskell}
e1 = unsafeResults $ sort <#>
      (convert [1..20::Integer]) >>= convert :: [[Integer]]
\end{minted}

\end{document}
