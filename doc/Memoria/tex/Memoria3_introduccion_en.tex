\documentclass[class=article, crop=false]{standalone}
\usepackage[utf8]{inputenc}
\usepackage[spanish]{babel}
\usepackage{minted}
\usepackage{xcolor}

\definecolor{bg}{rgb}{0.95,0.95,0.95}

\begin{document}

\section{Introduction}

This document is part of the result of a academic year of work to complete the BSc Thesis for
the Computer Science Engineering degree at the Complutense University of Madrid. In this
document we tackle the proposal and implementation of a small and simple functional language
with standard features in functional languages such as higher order and pattern matching, to
which is added another one not usual in this paradigm, but in an extension of it, the
so-called functional logic programming \cite{antoy2010functional}, which combines features of
independent paradigms of logical and functional programming. From the first it adopts in
particular the possibility of indeterministic computations that are sustained in the most
well-known languages of the paradigm such as Curry \cite{Hanus16Curry} and Toy
\cite{fraguas1999toy}, through the notion of non-deterministic function, that is to say, a
function that for given arguments can return more than one value.

In classical works such as \cite{hussmann1993nondeterminism} there is a relationship between
two fundamental types of semantics for indeterminism in functional or equational programming:
semantic of no sharing (commonly known as `run-time choice') or semantic of sharing
(call-time choice). In the section \ref{sec:prog_func_ind} we explain the difference between
the two types.

The standard approach to functional logic programming, whose theoretical foundations are
found in \cite{DBLP:journals/jlp/Gonzalez-MorenoHLR99} has considered that the most suitable
semantics for practical programming is call-time choice, which is why it is adopted in
languages such as Curry or Toy.

The traditional implementations of these languages have been based on compilation to Prolog,
that is, the original programs are translated to object Prolog programs, so that the Prolog
execution of these programs already expresses correctly the semantics of the original
programs. This trend changed in KiCS2 \cite{BrasselHanusPeemoellerReck11}, a Curry
implementation based on Haskell, in which the mode that captures the semantics of call-time
choice is quite dark and dependent on low-level aspects. Later, S. Fischer made another
purely functional proposal for call-time choice semantics with lazy
evaluation \cite{fischer2011purely} that is much clearer and more abstract. That proposal was
embodied in a Haskell library which, with the necessary adaptations, has been an important
basis for this work.

The task of combining these two semantics is developed independently in other works
\cite{riesco2014singular}. However, these proposals were only implemented as 'patches' in the
Toy system whose implementation is based on Prolog.

The essential part of this work is to create a purely functional programming language where
you can see the potential of the combination of the two types of semantics and what it can
mean for the description of indeterministic algorithms. There are practical examples where
neither of the two semantics adapts well to solve a problem but a combination of both ones. I
will try to translate the utility of this combination in the examples used throughout the
document.

\subsection{Programming with no-deterministic functions}

Non-deterministic programming has repeatedly shown to simplify the way in which an algorithm
can be written, such as in languages such as Prolog, Curry and Toy. People like Antony and
Hanus think this is because the results can be seen individually rather than as elements
belonging to a set of possible results \cite{antoy2010functional}. A recurrent example to
explain non-deterministic programming is usually the indeterministic result of a coin
flipped, heads or cross, 0 or 1, and then proceed to explain the ways in which an expression
with this deterministic value can be understood, how a sum of these values would work, or how
a call to a function would work. If that piece appears multiple times, must all of them have
the same value or can they have different values? Traditionally in non-deterministic
programming there are two types of semantics that tell us how to interpret these
non-deterministic pieces. These two semantics are the 'Call-time choice' and the 'run-time
choice'.

\subsubsection{Call-time choice}

Conceptually, 'call-time choice' semantics consists of choosing the non deterministic value
of each of the arguments of a function before applying them \cite{fischer2011purely}. One of
these languages is Curry. Let's see an example of this behavior:

\begin{minted}[bgcolor=bg]{haskell}
choose x _ = x
choose _ y = y

coin = choose 0 1

double x = x + x
\end{minted}

Let's first comment on the definition of \verb`choose`. If we write that definition in
Haskell it will always choose the first equation, but in Curry you can express
non-determinism by writing several overlapping definitions. In this way, \verb`choose 0 1`
can be reduced to \verb`0` or a \verb`1`, this is exactly non-determinism! Now that
I've explained the concept of non-determinism, let's see what the 'call-time choice' is. In
the statement \verb`double x = x + x`, \verb`x` appears as an argument. This means that all
appearances of \verb`x` will correspond to the same value. If we evaluate \verb`double coin`
we will get only two results, \verb`0` and \verb`2`. If we simply evaluate the expression 
\verb`coin + coin` in the interpreter we will get four results, \verb`0, 1, 1, 2`. That is to
say, in the semantics of `call-time choice' it is as if the values of the arguments are
chosen in the call of the function and `share' it in all the expression.

\subsubsection{Run-time choice}
The semantics `run-time choice' is much simpler. The values of non deterministic parts are
not shared. Each appearance of a non-deterministic piece will generate all the results that
correspond to it. In a language with these characteristics the evaluation of
\verb`double coin` of the previous example will produce the four results.

That is to say, in this semantics it must be understood that each non deterministic piece
will act as an independent source of indeterministic values, unlike in `call-time choice'
where the appearances of the arguments of a function are linked to the same result throughout
the entire expression.

\subsubsection{Other semantics}
There are other semantics in the field of indeterministic programming, such as the one
specified in the work \textit{Singular and plural functions for functional logic programming}
\cite{riesco2014singular}, which is proposed by the lack of concision of the semantics
`run-time choice' when pattern matching comes into play.

Another semantics is indeed the combination of the two types of semantics mentioned above,
which is the core of this work. In our work the combination is achieved with primitives
present in Sharade, which specify when to share the value of a variable in all its
apparitions throughout an expression. Later we will give examples of the use of this
semantics.

\subsection{Objectives}
My purpose in this BSc Thesis is to create a functional language that combines the
two previous semantics. To do this, the language will have primitives to specify when
nondeterministic pieces of an expression must share the same value and when they must
behave in the same way as in run-time choice. This language will have characteristics of a
modern functional language, such as higher order functions, lazy evaluation, lambda
expressions, etc. All this combined with non-determinism, for example, a part in an
expression can correspond to an indeterministic function. Everything will be implemented in
Haskell taking advantage of features it already has as standard in addition to a series of
libraries. The language is simple for a matter of dimensions of the work and to show the the
core of it, that is, the combination of the two semantics. Efficiency is not pursued, but in
the section \ref{sec:rendimiento} the performance is discussed and some ideas are given for
improve it.

To combine the two types of semantics, Sharade has a primitive structure
\verb`choose ... in ...`. It's like a \verb`let ... in ...`, but it has a behavior in which
the bound identifier that is created on the left side will share the value in all occurrences
of that identifier on the right side. If nothing is specified, my language semantically
behaves the same as `run-time choice'. For example:

\begin{minted}[bgcolor=bg]{haskell}
coin = 0 ? 1 ;

f x = x + x ;
f' x = choose x' = x in x' + x' ;
\end{minted}

If we evaluate \verb`f coin` we get the four values from the previous example, but if we
evaluate \verb`f' coin` we get two values, because a binding for \verb`x'` has been created.
This way, all occurrences of \verb`x'` will share the same value when evaluating the
expression. To clarify, the operator infixed \verb`?` is a primitive of Sharade, it means the
same as the function \verb`choose` of the previous example.

In conclusion, Sharade combines the two types of semantics, letting the user specify which
pieces will behave in a `call-time choice' regime. This combination of the two semantics
has not been seen in existing literature. It is true that there is a similar approach in
\textit{A Flexible Framework for Programming with Non-deterministic Functions}
\cite{lopez2009flexible}, but it is not as clear or explicit as the one I
propose in Sharade.

\subsection{Workplan}
First we will try to express the indeterminism in Haskell. For this we will need a certain
fluency in monadic programming.  The indeterminism in Haskell by default is dealt with the
semantics of `run-time choice'. Every time you want to `extract' a value from an
indeterministic (monadic) expression, all the results will appear. This type of programming
is in itself quite tedious, although elegant abstractions such as notation \verb`do` can be
used.

Then we'll understand Fischer's explicit-sharing library, which brings the `call-time choice'
features to Haskell without losing lazy evaluation. He achieves this thanks to the state
monads, with which he implements the early choice and late evaluation. In other words, the
choices of the value to use are made at the moment but the evaluation of that value is only
done when it is required. If the programming was already tedious before, now it is even more,
since a higher level of monads is added, making necessary more extractions of values of the
monads, more lines of code.

It turns out then the convenienve of creating a small language where all the syntax is
discharged, but wide enough to express what concerns us, the combination of semantics and
their usefulness. With a behavior of `run-time choice' by default and at the request of the
programmer, `call-time choice' on a specified variable.

With all this in mind, it is necessary to create a Sharade program compiler to Haskell
programs, with a translation function being part of the heart of the work. This translation
function is discussed in the section \ref{sec:traduccion}.

\subsection{Document structure}
This document is intended for people with minimal Haskell knowledge even if they don't know
what non-deterministic programming is. To this end, at this introductory point we have
explained what this paradigm is about with existing languages such as Curry. Later we will
explain how Sharade approaches this paradigm with the combination of the two types of
semantics that concern us. The rest of the paper deals with the implementation of Sharade.
Due to the complexity underlying the implementation of this paradigm in Haskell, it is
necessary to explain quite a few monadic programming concepts in section \ref{sec:con_prev}.
Then we will comment on the implementation of Fischer that solves the problem of sharing with
lazy evaluation in the section \ref{sec:explicit_sharing}. In point \ref{sec:parsec} we will
make a brief explanation about the library \verb`Parsec` that we use to perform the Sharade
syntax analysis. Once the reader is comfortable with the concepts of monadic programming,
indeterminism and the two semantics, we will approach the implementation of Sharade, from
parsing, type analysis and translation. Finally there will be a section of conclusions, where
I will collect some comments on performance, how to improve the implementation of the
project, how new features can be incorporated into the language, such as user types, other
ways to collect results and possible future work.

\end{document}
