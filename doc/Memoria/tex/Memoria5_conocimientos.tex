\documentclass[class=article, crop=false]{standalone}
\usepackage[utf8]{inputenc}
\usepackage[spanish]{babel}
\usepackage{minted}
\usepackage{xcolor}

\definecolor{bg}{rgb}{0.95,0.95,0.95}

\begin{document}

\section{Conocimientos previos}\label{sec:con_prev}
En este apartado explicaremos conceptos que son necesarios para un buen entendimiento de este
trabajo, ya que hacemos uso de una programación monádica muy intensiva. Creemos que un
programador aficionado no tiene por qué conocer los detalles ni tener soltura implementando
instancias de las clases de tipos que veremos, por lo que intentaremos hacer una progresión
de menor a mayor dificultad. Algún ejemplo puede resultar un reto entenderlo, sobre todo si
nunca se ha tratado con este tipo de programación con anterioridad.

Muchas de las explicaciones de este punto están remotamente inspiradas en el libro\textit{
Learn You a Haskell for Great Good} \cite{Lipovaca:2011:LYH:2018642}. El código de ejemplo y
las explicaciones son de nuestro puño y letra, pero hemos tomado prestadas algunas
comparaciones como `caja' para poder hacer la explicación mas entendible para las personas
que no están acostumbradas al paradigma de la programación funcional.

\subsection{Clases de tipos}
Las clases de tipos en Haskell son agrupaciones de tipos que tienen la característica común
de que implementan una serie de funciones. Por ejemplo, \verb`Int` o \verb`Double` pertenecen
a la clase de tipos \verb`Num`, tienen en común que implementan las funciones suma,
multiplicación, resta, valor absoluto...

Dicho de otra manera, las clases de tipos son una especie de interfaz que implementa algún
tipo concreto, añadiéndole alguna funcionalidad. No hay que confundir las clases de tipo con
las clases de Java o C++, no tiene nada que ver. Una comparación mas idónea sería decir que
son como las interfaces de Java, pero en nuestra opinión las clases de tipos son más
potentes.

\subsection{Familias de tipos, \textit{Kinds}}
También es conveniente entender cómo funcionan los tipos en Haskell. Los tipos en Haskell
pertenecen a familias de tipos. La familia de \verb`Int` o de \verb`Double` o de \verb`[Int]`
es \verb`*`. Un asterisco quiere decir que es un tipo concreto. Por ejemplo, la familia de
\verb`[]`, las listas, es \verb`* -> *`, es decir, necesita un tipo concreto para dar otro
tipo concreto. De este modo, la familia de \verb`Either` es \verb`* -> * -> *`, necesita dos
tipos concretos para dar otro. Es parecido al tipo de las funciones, donde \verb`Int -> Int`
representa un tipo funcional que espera un \verb`Int` para dar un \verb`Int`, solo que en
este caso solo existe un tipo básico, \verb`*`. Se puede ver que esta semántica no tiene la
misma riqueza que la semántica de los tipos de Haskell, donde existen múltiples variables de
tipos (\verb`a`, \verb`b`...) al contrario que esta, donde solo tenemos el asterisco.

Hay que remarcar que no se debe confundir las familias de tipos con las clases de tipos. Una
clase de tipos es por ejemplo \verb`Num`, \verb`Fractional`. Rizando el rizo, las siguientes
son clases de constructoras de tipos. Se llaman así porque, por ejemplo, se puede entender
que \verb`[]` es una constructora de tipos, igual que \verb`Either`. Así, en vez de ser
simples clases de tipos, se convierten en clases de constructoras de tipos.

Podríamos dar una vuelta de tuerca más y explicar que las familias de tipos también tienen
su sistema de tipos especial, ya que esperan una familia de tipos concreta, pero no es el
objetivo de este trabajo por lo que no creo que tengan mucho interés. Eso sin contar que,
hasta donde llegan mis conocimientos de Haskell, no hay sistemas superiores de tipos que
esperen como argumentos familias de tipos.

\subsection{Funtores}
Los funtores, \verb`Functor` en inglés, es una clase de constructora de tipos de Haskell
definida de la siguiente forma:

\begin{minted}[bgcolor=bg]{haskell}
class Functor (f :: * -> *) where
  fmap :: (a -> b) -> f a -> f b
\end{minted}

Define una función \verb`fmap` que necesita una función que mueva elementos del tipo \verb`a`
a elementos del tipo \verb`b`, un valor de tipo \verb`f a` y devuelve un \verb`f b`. Como se
puede ver, \verb`f` es un tipo que pertenece a la familia \verb`* -> *`, es decir, necesita
un tipo para ser un tipo concreto. Hay muchas constructoras de tipos que pertenecen a esta
clase, como la del tipo de las listas o la constructora de tipos \verb`Maybe`.

\begin{minted}[bgcolor=bg]{haskell}
instance Functor [] where
  fmap = map

instance Functor Maybe where
  fmap _ Nothing = Nothing
  fmap f (Just x) = Just (f x)
\end{minted}

Efectivamente, la función \verb`fmap` para las listas es exactamente la función \verb`map`,
que en este caso, el funtor \verb`f` es \verb`[]`. Conceptualmente, \verb`map` hace lo que
tiene que hacer \verb`fmap`, mover valores dentro de un funtor de un tipo a otro, dicho de
otra manera, transformar una lista de elementos de \verb`a` a otra lista de elementos de
\verb`b`.

Se puede entender los tipos de esta clase de tipos como cajas que encierran un tipo. Por
ejemplo, \verb`[]` es una caja que encierra un tipo, \verb`Maybe` también. Esta analogía de
las cajas es bastante usada en la literatura, y la usaré más veces explicando las demás
clases de tipos.

Otro ejemplo sería el caso de \verb`Either a b`. No se puede conseguir meter este tipo en la
clase \verb`Functor` con los dos valores, ya que es obligado que pertenezca a la familia de
tipos \verb`* -> *`. Pero como también existe aplicación parcial en las constructoras de
tipos, se puede meter el tipo \verb`(Either a)`' en la clase \verb`Functor`:

\begin{minted}[bgcolor=bg]{haskell}
instance (Either a) where
  fmap _ (Left a)  = Left a
  fmap f (Right b) = Right (f b)
\end{minted}

\subsection{Funtores aplicativos}
Podemos `bajar' de alguna manera en la jerarquía de las clases de tipos, encontrándonos con
los funtores aplicativos. Se llaman aplicativos porque es un funtor que se puede aplicar,
es decir, la función se recibe dentro de una `caja', ¡dentro de un funtor! El valor también
se recibe dentro de una caja. Veamos cómo está declarada esta clase:

\begin{minted}[bgcolor=bg]{haskell}
class Functor f => Applicative (f :: * -> *) where
  pure :: a -> f a
  (<*>) :: f (a -> b) -> f a -> f b
\end{minted}

Antes de estar en la clase de los funtores aplicativos, un requisito es ser un funtor. Para
terminar de serlo, hay que tener implementadas dos funciones, \verb`pure` y el operador
infijo \verb`<*>`. \verb`pure` es simplemente una función que recibe un valor tradicional y
lo devuelve metido en una caja. El operador \verb`(<*>)`, primo hermano de \verb`fmap`,
recibe una función metida en una caja, un valor metido en una caja y devuelve un valor
encerrado en una caja aplicando la función. Veamos cómo se implementa la pertenencia a esta
clase para algunos tipos:

\begin{minted}[bgcolor=bg]{haskell}
instance Applicative Maybe where
  pure x = Just x
  (<*>) Nothing _ = Nothing
  (<*>) _ Nothing = Nothing
  (<*>) (Just f) (Just x) = Just (f x)
  --(Just f) <*> x = fmap f x -- Otra implementación posible

instance Applicative [] where
  pure x = [x]
  fs <*> xs = concatMap (\f -> map f xs) fs

instance Applicative ((->) c) where
  pure x = (\c -> x)
  f <*> f' = (\c -> f c (f' c))
\end{minted}

El más sencillo es el primero, la implementación de \verb`Maybe`. Si la función o el valor
son Nothing, el resultado es Nothing, ya que nos faltan argumentos con los que trabajar. Si
tenemos los valores, los sacamos de sus respectivas cajas, aplicamos la función y volvemos a
meter en una caja el resultado.

La lista es más complicada. Por la izquierda, \verb`fs`, recibimos una lista de funciones,
por la derecha, \verb`xs`, una lista de valores. Tenemos que aplicar cada función a todos los
valores. Para ello hacemos un \verb`map` con cada función y luego concatenamos todas las
listas generadas.

El ejemplo más complicado, si duda, es el tercero. \verb`(->)` es una constructora primitiva
de Haskell que crea tipos funcionales. Necesita dos tipos, el tipo origen y el tipo destino.
Por ejemplo, \verb`(+) :: Int -> (Int -> Int)`. Así, \verb`((->) c)` es una aplicación
parcial de la constructora de tipos. De este modo, su familia de tipos es \verb`* -> *`, por
lo que es apto para ser un funtor aplicativo. De la declaración de tipo de \verb`(<*>)` se
deduce que \verb`f :: c -> a -> b`, \verb`f' :: c -> a`, por lo tanto el tipo del resultado
debe ser \verb`c -> b`, una función. Así, primero aplicamos \verb`f' c :: a`, después
aplicamos \verb`((f c :: a -> b) (f' c :: a))`, que unido a la lambda expresión que engloba a
todo, se obtiene el tipo \verb`c -> b`, como se quería. Visto así puede ser complicado,
veamos un ejemplo no muy práctico con números, donde \verb`a, b, c :: Int`:

\verb`((+) <*> (\a -> a + 10)) 3` \\
\verb`= (\c -> (c+) ((\a -> a + 10) c) 3` \\
\verb`= (\c -> (c+) (c+10)) 3` \\
\verb`= (3+) (13) = 16`

Este ejemplo viene bien como entrenamiento para lo que viene, ya que en las mónadas con
estado se usan funciones para `esconder' valores, aprovechando las características de
memoización de Haskell, de la misma manera que en un lenguaje funcional sin evaluación
perezosa se puede conseguir el mismo efecto pasando funciones. En el ejemplo, creamos una
lambda para recuperar el valor \verb`c`.

\subsection{Mónadas}
Podemos seguir bajando en la jerarquía de clases de tipos para encontrarnos con las mónadas.
Para seguir con la casuística de meter cosas en cajas, ahora la función no está metida en una
caja... ¡pero devuelve un valor metido en una caja! Veamos la declaración de la clase:

\begin{minted}[bgcolor=bg]{haskell}
class Applicative m => Monad (m :: * -> *) where
  return :: a -> m a
  (>>=) :: m a -> (a -> m b) -> m b
\end{minted}

Como antes, para ser una mónada es requisito indispensable ser un funtor aplicativo. Si antes
había que implementar la función prima hermana de \verb`fmap`, ahora hay que implementar la
primo-segunda, \verb`(>>=)`. Ahora la función recibe un valor metido en una caja y una
función que recibe un valor y devuelve otro metido en una caja. Devuelve esa caja intacta. La
función \verb`return` es exactamente igual que \verb`pure`. Veamos algunas implementaciones:

\begin{minted}[bgcolor=bg]{haskell}
instance Monad Maybe where
  return x = pure
  Nothing >>= _  = Nothing
  (Just x) >>= f = f x

instance Monad [] where
  return x = pure
  [] >>= _ = []
  (x:xs) >>= f = f x ++ xs >>= f

instance Monad (Either a) where
  return = pure
  (Left l) >>= _  = Left l
  (Right r) >>= f = f r
\end{minted}

Las mónadas permiten, entre otras cosas, escribir bibliotecas modulares y trabajar de una
forma elegante con errores en Haskell. Por ejemplo, la implementación de una pila usando
Maybe para devolver Nothing en caso de errores:

\begin{minted}[bgcolor=bg]{haskell}
data Stack a = Stack [a]

emptyStack :: Maybe (Stack a)
emptyStack = Just (Stack [])

push :: a -> Stack a -> Maybe (Stack a)
push x (Stack xs) = Just (Stack (x:xs))

pop :: Stack a -> Maybe (Stack a)
pop (Stack []) = fail "Empty Stack" -- Generación de error
pop (Stack (x:xs)) = Just (Stack xs)

top :: Stack a -> Maybe a
top (Stack []) = fail "Empty Stack" -- Generación de error
top (Stack (x:xs)) = Just x

ejemplo :: Maybe (Stack Int)
ejemplo = emptyStack >>= push 1 >>= push 2 >>= pop >>= pop
\end{minted}

Ha aparecido una nueva función, \verb`fail`. Esta función recibe un String y devuelve un
valor monádico. Con la mónada \verb`Maybe`, esta función simplemente devuelve \verb`Nothing`,
pero otras mónadas más complejas podrían llevar la traza del error. Por ejemplo, en la
biblioteca \verb`parsec` se sigue esta filosofía. Los errores y el estado en en análisis se
llevan a cabo con una unión de mónadas (que forman una mónada), cada una ocupándose de su
tarea. Lo ideal es adaptar el código para que use cualquier tipo de mónada, dando al usuario
de tu biblioteca una gran flexibilidad y facilidad para usar tu código en cualquier parte. La
biblioteca \verb`parsec` también permite esto, bajo ciertas condiciones, lo que permite que
se desarrollen ciertas funcionalidades encima de ésta que pueden facilitar mucho el análisis.
En concreto, se usan las \verb`MonadTransformer`, que permiten la composición de mónadas.

Otro aspecto secundario (pues es mera sintaxis) pero importante en la programación monádica
en Haskell es la introducción de la notación \verb`do`, que permite de forma elegante traer
la programación imperativa a Haskell. Así, otra posible función ejemplo para nuestro código
podría ser:

\begin{minted}[bgcolor=bg]{haskell}
ejemplo :: Stack Int -> Maybe (Stack Int)
ejemplo stack = do
  valor1 <- top stack
  stack  <- pop stack
  valor2 <- top stack
  stack  <- pop stack
  push (valor1 + valor2) stack
\end{minted}

Saca dos valores de la pila, los suma y deposita el resultado. Para terminar, la notación
\verb`do` es solo azúcar sintáctico: \\
La expresión \\
\verb`do { v1 <- e1 ; v2 <- e2 ; en }` \\
se transforma en \\
\verb`e1 >>= (\v1 -> e2 >>= (\v2 -> en))`.

\subsection{Mónadas 'Plus'}
Como dice el nombre, las mónadas `plus' no son más que mónadas ampliadas con dos simples
operaciones. Estas dos operaciones serán muy útiles para la programación no determinista que
tenemos entre manos. Veamos cómo está definida:

\begin{minted}[bgcolor=bg]{haskell}
class (Alternative m, Monad m) => MonadPlus (m :: * -> *) where
  mzero :: m a
  mplus :: m a -> m a -> m a
\end{minted}

No hay que prestar especial atención a la clase Alternative, es, en casi todos los casos,
exactamente igual que esta clase. Para estar en la clase \verb`MonadPlus`, hay que estar en
la clase Monad e implementar dos operaciones. La primera operación, \verb`mzero`, debe ser
una función que retorne un valor vacío, por defecto, para esa mónada. La función \verb`mplus`
debe ser una función que tome dos valores monádicos y devuelva otro que represente la unión
de los dos primeros. La implementación para algunos tipos es como sigue:

\begin{minted}[bgcolor=bg]{haskell}
instance MonadPlus [] where
  mzero = []
  mplus a b = a ++ b

instance MonadPlus Maybe where
  mzero = Nothing
  mplus Nothing b = b
  mplus a Nothing = a
  mplus a b = a
\end{minted}

Usaremos estas operaciones para implementar algunas características de la programación no
determinista, por ejemplo, la elección de dos valores. \verb`a?b` en Sharade se traducirá
como \\ \verb`mplus (return a) (return b)`, el cómputo vacío, cuando no hay ningún resultado,
se representará con \verb`mzero`. Esta situación se puede dar en una expresión \verb`case`
cuando no se encaja en ningún patrón. Haskell en esos casos hace saltar una excepción. En la
biblioteca explicit-sharing se extiende además la clase MonadPlus para permitir otra
operación, \verb`share`, para permitir que en una expresión el valor de una variable se
comparta entre todas las apariciones de esa variable. Más adelante veremos cual es la mónada
que se usa para evaluar estos cómputos no deterministas.

\subsection{Mónadas con estado}
Podemos pensar las mónadas con estado como mónadas normales pero que además de un valor
tienen un estado `oculto'. Veamos como se define la clase y su implementación para el tipo
\verb`State`:

\begin{minted}[bgcolor=bg]{haskell}
class Monad m => MonadState s (m :: * -> *) | m -> s where
  get :: m s
  put :: s -> m ()
\end{minted}

La función \verb`get` obtiene el estado oculto y lo pone como valor, así lo podremos
obtener fácilmente en la notación \verb`do`. La función \verb`put` recibe un valor del tipo
\verb`s` (el tipo del estado), lo oculta y encierra el valor \verb`()`. Vamos a ver cómo se
implementa la pertenencia a esta clase para un tipo concreto:

\begin{minted}[bgcolor=bg]{haskell}
{-# LANGUAGE FlexibleInstances, MultiParamTypeClasses #-}
import Control.Monad.State hiding (State)

data State s a = State { runState :: s -> (a, s) }
\end{minted}

Como anotación, \verb`FlexibleInstances` y \verb`MultiParamTypeClasses` son dos extensiones
del lenguaje de las muchas que hay en Haskell. \verb`FlexibleInstances` permite que las
instancias de las clases no tengan que ser obligatoriamente un constructor de tipos y
opcionalmente una lista de variables de tipos. Por ejemplo, la instancia \verb`Num (Maybe a)`
está permitida pero no \verb`Num (Maybe Int)`, ya que \verb`Int` no es una variable de tipo.
La segunda extensión permite que las clases de tipos reciban varios argumentos, como
\verb`MonadState`. Así, gracias a estas dos extensiones se puede implementar la instancia
\verb`MonadState s (State s)`. La primera permite que aparezca \verb`State s`, ya que sin
ella la variable de tipos \verb`s` no puede volver a aparecer y la segunda permite que haya
más de un argumento para la clase de tipos.

Esta es la definición de nuestro tipo. Nuestro tipo \verb`State` encierra un estado y un
valor en una función. Esta función tiene como objetivo hacer las modificaciones necesarias al
estado. Puede ser tan sencillo como dejar el estado intacto tal y como lo recibe. El valor
tiene que ser dado desde fuera, igual que es dado desde fuera en Maybe, las listas...

\begin{minted}[bgcolor=bg]{haskell}
instance Functor (State s) where
  -- f :: a -> b
  -- sf :: s -> (a, s)
  -- s, s2 :: s
  -- v :: a
  fmap f (State sf) = State (\s ->
    let (v, s2) = sf s
    in (f v, s2))
\end{minted}

\verb`fmap` está implementada de la siguiente forma. Se obtiene la función de dentro del
\verb`State` que se recibe, \verb`sf` y se crea un nuevo \verb`State` con una nueva función.
Esta función, como todas, recibe un estado, se aplica \verb`sf` a ese estado, se obtiene un
valor y un nuevo estado. Tenemos que transformar el valor obtenido con la función \verb`f` y
devolverlo conjunto al nuevo estado generado.

\begin{minted}[bgcolor=bg]{haskell}
instance Applicative (State s) where
  pure a = State (\s -> (a, s))

  -- f  :: s ->  (a -> b, s)
  -- sf :: s -> (a, s)
  -- f' :: a -> b
  -- v2 :: a
  -- s, s2, s3 :: s
  (State f) <*> (State sf) = State (\s ->
    let (f', s2) = f s
        (v2, s3) = sf s2
        in (f' v2, s3))
\end{minted}

En este caso, recibimos una función dentro de un \verb`State`. Tenemos que hacer lo mismo de
antes, aplicar la función al valor de la derecha, pero con más cuidado. Para obtener la
función, aplicamos \verb`f` a \verb`s`. Una vez conseguida la función, hacemos lo mismo que
\verb`fmap`, ¡incluso podríamos copiar y pegar! Si no fuera por los nombres de las
variables...

\begin{minted}[bgcolor=bg]{haskell}  
instance Monad (State s) where
  return = pure

  -- sf :: s -> (a, s)
  -- f :: a -> State s b
  -- s, s1 :: s
  -- v :: a
  -- f' :: s -> (b, s)
  (State sf) >>= f = State (\s ->
    let (v, s1) = sf s
        State f' = f v
        in f' s1)
\end{minted}

Puede que este sea más sencillo. Como siempre, tenemos que aplicar la función al estado que
viene. Después, aplicamos la función al valor obtenido, obteniendo de nuevo un \verb`State`.
Finalmente aplicamos la nueva función al estado anterior.

\begin{minted}[bgcolor=bg]{haskell}  
instance MonadState s (State s) where
  get = State (\s -> (s,s))
  put s = State (\_ -> ((),s))
\end{minted}

Implementar la nueva clase es más sencillo que todas las demás. Recordemos que \verb`get`
tenía que poner el estado como valor sin mutarlo. Recordemos también que \verb`put` tenía que
poner el estado que se le pasaba como argumento y vaciar el valor.

Veamos un ejemplo. En este ejemplo, el estado es una lista, representando una pila. El valor
puede ir cambiando de tipo a lo largo del programa, como con cualquier otro tipo de mónada:

\begin{minted}[bgcolor=bg]{haskell}  
push :: a -> State [a] ()
push x = do
  xs <- get
  put (x:xs)

pop :: State [a] a
pop = do
  (x:xs) <- get
  put xs
  return x

ejemplo :: State [Int] Int
ejemplo = do
  -- ¡`pop` está actuando sobre 0 argumentos! Está usando el
  -- estado oculto, gracias a las funciones `get` y `put`.
  v1 <- pop
  v2 <- pop
  push (v1 + v2)
  pop

startState = [1, 2]
main = print $ runState ejemplo startState
\end{minted}

Si evaluamos main, se muestra por pantalla \verb`(3,[])`'. El primer valor de la pareja es el
resultado de nuestro programa y el segundo, \verb`[]` es el resultado del estado interno de
nuestro programa. El estado inicial empezó con dos valores, se extrajeron, se sumaron, se
depositó el resultado, y a modo de ejemplo, se sacó el resultado. Como se puede ver,
\verb`pop` tiene cero argumentos, la función \verb`ejemplo` tampoco tiene, ¡\verb`pop` está
cogiendo los valores del estado interno del programa! Por este motivo, y entre otros, las
mónadas con estado son tan potentes. Permiten escribir código Haskell que dependen de un
estado que a priori está oculto.

Las \verb`MonadState` tienen una fuerte presencia en este proyecto. Se encargan de
implementar de una manera eficiente la compartición de valores no deterministas en mi
lenguaje, como explicaré en el siguiente punto. También se encargan de una parte importante
del análisis sintáctico. En ese caso, el estado actúa como la cadena de caracteres que faltan
por reconocer y los valores van cambiando dependiendo de la estructura sintáctica. Unas veces
es una expresión, otras un número y otras el programa entero (y esperemos que ahí el estado
sea la cadena vacía, si no hay un error sintáctico, pero de eso se encarga otra mónada).

A modo de nota final, nos gustaría aclarar que el tipo \verb`State` como tal no existe
en Haskell, realmente es un alias del tipo \verb`StateT s Identity a`. \verb`StateT` es una
mónada transformadora. Como ya hemos explicado de manera superficial en otros puntos,
permiten componer varias mónadas distintas, para conseguir un buen desacoplamiento de las
tareas que hay que hacer en el programa.

\end{document}
