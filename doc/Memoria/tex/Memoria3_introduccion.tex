\documentclass[class=article, crop=false]{standalone}
\usepackage[utf8]{inputenc}
\usepackage[spanish]{babel}
\usepackage{minted}
\usepackage{xcolor}

\definecolor{bg}{rgb}{0.95,0.95,0.95}

\begin{document}

\section{Introducción}
Este documento es parte del resultado de un año lectivo de trabajo para concluir el trabajo
de fin de grado de la carrera de Ingeniería Informática en la Universidad Complutense de
Madrid. En este documento abordo la propuesta e implementación de un pequeño y simple
lenguaje funcional con características habituales en lenguajes funcionales como el orden
superior y ajuste de patrones, a las que se añade otra no habitual en este paradigma, pero sí
en una extensión de él, la denominada programación lógico
funcional \cite{antoy2010functional}, que combina características de los paradigmas
independientes de programación lógica y funcional. Del primero adopta en particular la
posibilidad de cómputos indeterministas que se encuentran sustentados en los lenguajes más
conocidos del paradigma como Curry \cite{Hanus16Curry} y Toy \cite{fraguas1999toy}, en la
noción de función indeterminista, es decir, función que para unos argumentos dados puede
devolver más de un valor.

En trabajos clásicos \cite{hussmann1993nondeterminism} se distingue entre dos tipos
fundamentales de  semántica para el indeterminismo en programas funcionales o ecuacionales en
general: semántica de no compartición (conocida habitualmente como `run-time choice') o
semántica de compartición (call-time choice). En el apartado \ref{sec:prog_func_ind}
explicamos la diferencia entre ambas.

La corriente estándar de la programación lógico funcional, cuyos fundamentos teóricos se
encuentran en \cite{DBLP:journals/jlp/Gonzalez-MorenoHLR99} ha considerado que la semántica
más adecuada para la programación práctica es la de call-time choice, que es por ese motivo
la adoptada en lenguajes como Curry o Toy.

Las implementaciones tradicionales de estos lenguajes han estado basadas en compilación a
Prolog \cite{deransart2012prolog}, es decir, los programas originales se traducen a programas
Prolog objeto, de modo que la ejecución Prolog de estos ya expresa correctamente la semántica
de los programas originales. Esta tendencia cambió en KiCS2
\cite{BrasselHanusPeemoellerReck11}, una implementación de Curry basada en Haskell, en la que
el modo en el que se captura la semántica de call-time choice resulta bastante oscuro y
dependiente de aspectos de bajo nivel. Más adelante, S. Fischer realizó otra propuesta
puramente funcional para la semántica de call-time choice con evaluación
perezosa \cite{fischer2011purely} que resulta mucho más clara y abstracta. Esta propuesta
quedó plasmada en una biblioteca de Haskell, \verb`explicit-sharing` que, con las
adaptaciones necesarias, ha constituido una base importante de este trabajo.

De manera independiente, en otros trabajos \cite{riesco2014singular} se desarrolla la tarea
de combinar las dos semánticas citadas. Sin embargo, esas propuestas solo se implementaron
como `parches' muy poco abstractos en el sistema Toy cuya implementación está basada en
Prolog.

La parte esencial del trabajo que presentamos aquí es crear un lenguaje puramente funcional
donde se vea el potencial de la combinación de los dos tipos de semánticas y lo que ello
puede suponer para la descripción de algoritmos indeterministas. Hay ejemplos prácticos donde
ninguna de las dos semánticas se adapta bien para resolver un problema pero sí una
combinación de ambas. Intentaré plasmar la utilidad de esta combinación en ejemplos de uso a
lo largo de este documento.

\subsection{Programación con funciones no deterministas}\label{sec:prog_func_ind}
La programación no determinista ha demostrado en varias ocasiones simplificar la forma en la
que un algoritmo puede ser escrito, como por ejemplo en lenguajes como Prolog, Curry y Toy.
En \cite{antoy2010functional} se alega que esto es debido a que los resultados pueden ser
vistos individualmente en vez de como elementos pertenecientes a un conjunto de posibles
resultados. Un ejemplo recurrente para explicar algunos aspectos básicos de la programación
no determinista suele ser el resultado indeterminista de una moneda lanzada al aire, cara o
cruz, 0 o 1, para luego proceder a explicar las formas de las que se puede entender una
expresión con este valor determinista, cómo funcionaría una suma de estos valores o cómo
funcionaría una llamada a una función. Si esa pieza aparece múltiples veces ¿deben tener
todas ellas el mismo valor o pueden tener distintos? Tradicionalmente en la programación no
determinista existen dos tipos de semántica que nos dicen como interpretar estas piezas no
deterministas. Estas dos semánticas son denominadas `Call-time choice' y `run-time choice'.

\subsubsection{Call-time choice}
Conceptualmente, la semántica `call-time choice' consiste en elegir el valor no determinista
de cada uno de los argumentos de una función antes de aplicarlos \cite{fischer2011purely}.
Uno de estos lenguajes es Curry. Veamos un ejemplo de este comportamiento:

\begin{minted}[bgcolor=bg]{haskell}
choose x _ = x
choose _ y = y

coin = choose 0 1

double x = x + x
\end{minted}

Primero vamos a comentar la definición de \verb`choose`. Si escribimos esa definición en
Haskell siempre escogerá la primera ecuación, pero en Curry se puede expresar el no
determinismo escribiendo varias definiciones que solapen. De este modo \verb`choose 0 1`
podrá reducirse a un \verb`0` o a un \verb`1`, ¡esto es justo el no determinismo! Ahora que
hemos explicado el concepto del no determinismo, veamos qué es el `call-time choice'. En la
declaración \verb`double x = x + x`, \verb`x` aparece como argumento. Esto significa que
todas las apariciones de \verb`x` corresponderán al mismo valor. Si evaluamos
\verb`double coin` obtendremos solamente dos resultados, \verb`0` y \verb`2`. Si por el
contrario simplemente evaluamos la expresión \verb`coin + coin` en el intérprete obtendremos
cuatro resultados, \verb`0, 1, 1, 2`. Es decir, en la semántica de `call-time choice' es como
si los valores de los argumentos se eligen en la llamada de la función y lo `comparten' en
toda la expresión.

\subsubsection{Run-time choice}
La semántica `run-time choice' es mucho más simple. Los valores de las piezas no
deterministas no se comparten. Cada aparición de una pieza no determinista generará todos los
resultados que le corresponden. En un lenguaje con estas características la evaluación de
\verb`double coin` del ejemplo anterior sí producirá los cuatro resultados.

Es decir, en esta semántica hay que entender que cada pieza no determinista, así como las
copias de esta que puedan aparecer por aplicación de las ecuaciones que definen las
funciones, actuará como una fuente independiente de valores indeterministas, al contrario que
en `call-time choice' donde las apariciones de los argumentos de una función están ligadas a
un mismo resultado a lo largo de toda la expresión.

\subsubsection{Otras semánticas}
Existen otras semánticas en el campo de la programación indeterminista, como la especificada
en el trabajo \cite{riesco2014singular}, que se propone por la falta de concisión de la
semántica `run-time choice' cuando el ajuste de patrones entran en juego.

Otra semántica es precisamente la combinación de los dos tipos de semánticas comentadas
anteriormente, considerada en \cite{riesco2014singular} y que es el corazón de este trabajo.
En nuestro trabajo la combinación se consigue con primitivas presentes en Sharade, que
especifican cuando compartir el valor de una variable en todas sus apariciones a lo largo de
una expresión. Posteriormente daremos ejemplos de uso de esta semántica.

\subsection{Objetivos}
Nuestro objetivo en este trabajo de fin de grado es crear un lenguaje funcional que combine
las dos semánticas anteriores. Para ello, el lenguaje tendrá primitivas para especificar
cuándo las piezas no deterministas de una expresión deben compartir el mismo valor y cuándo
se tienen que comportar de la misma forma que en run-time choice. Este lenguaje tendrá
características propias de un lenguaje funcional moderno, como orden superior, evaluación
perezosa, expresiones lambda, etc. Todo ello combinado con el no determinismo, por ejemplo,
una pieza en una expresión se puede corresponder con una función indeterminista. Todo estará
implementado en Haskell aprovechando características que ya tiene de serie además de una
serie de bibliotecas. El lenguaje es simple por una cuestión de acotación y para mostrar el
núcleo de este trabajo, la combinación de las dos semánticas. No se persigue la eficiencia,
pero en el apartado \ref{sec:rendimiento} se comenta el rendimiento y algunas ideas para
mejorarlo.

Para combinar los dos tipos de semánticas, Sharade tiene una estructura primitiva
\verb`choose ... in ...`. Es parecido a un \verb`let ... in ...`, pero tiene un
comportamiento en el que el identificador ligado que se crea en la parte de la izquierda
compartirá el valor en todas las apariciones de ese identificador en la parte derecha. Es
decir, tiene el mismo comportamiento que `call-time choice'. Si no se especifica nada, mi
lenguaje semánticamente se comporta igual que `run-time choice'. Por ejemplo:

\begin{minted}[bgcolor=bg]{haskell}
coin = 0 ? 1 ;

f x = x + x ;
f' x = choose x' = x in x' + x' ;
\end{minted}

Si evaluamos \verb`f coin` obtendremos los cuatro valores del ejemplo anterior, pero si
evaluamos \verb`f' coin` obtenemos dos valores, debido a que se ha creado una ligadura de
\verb`x'`. De esta forma, todas las apariciones de \verb`x'` compartirán el mismo valor al
evaluar la expresión. Para aclarar, el operador infijo \verb`?` es una primitiva más de
Sharade, significa lo mismo que la función \verb`choose` del ejemplo anterior.

En resumen, Sharade combina los dos tipos de semántica, dejando al usuario especificar qué
piezas se van a comportar como lo harían en `call-time choice'. Esta combinación de las dos
semánticas no lo hemos visto en la literatura ya existente, salvo en
\cite{lopez2009flexible}, pero allí no es tan clara ni explícita como la que yo propongo en
Sharade, ni se propone una implementación con principios claros.

\subsection{Plan de trabajo}
Primeramente intentaremos expresar el indeterminismo en Haskell. Para ello nos hará falta una
cierta soltura en programación monádica. El indeterminismo en Haskell por defecto se trata
con la semántica de `run-time choice'. Cada vez que se quiera `extraer' un valor de una
expresión indeterminista (monádica) surgirán todos los resultados. Este tipo de programación
ya es de por sí bastante farragosa, a pesar de que se pueden usar abstracciones bastante
elegantes como la notación \verb`do`.

Después comprenderemos la biblioteca \verb`explicit-sharing` de Fischer, que trae las
características de `call-time choice' a Haskell sin perder la evaluación perezosa. Esto lo
consigue gracias a las mónadas con estado, con las que implementa la elección temprana de un
valor no determinista y la evaluación tardía, que es exactamente lo que permite esta
evaluación perezosa. Es decir, las elecciones del valor a usar se hacen en el momento pero
la evaluación de dicho valor solamente cuando es necesaria. Si la programación ya era
farragosa antes, ahora mucho más, ya que se añade un nivel superior de mónadas, haciendo
necesarias más extracciones de valores de las mónadas, más líneas de código.

De aquí sale la necesidad de crear un pequeño lenguaje donde se descargue toda la sintaxis,
pero suficientemente amplio para expresar lo que nos atañe, la combinación de semánticas y su
utilidad. Con un comportamiento de `run-time choice' por defecto y a petición del
programador, `call-time choice' sobre una variable especificada.

Con todo esto en mente, es necesaria la creación de un compilador de programas Sharade a
programas Haskell, con una función de traducción formando parte del corazón del trabajo. Esta
función de traducción está comentada en el apartado \ref{sec:traduccion}.

\subsection{Estructura del documento}
Este documento está pensado para personas con un mínimo conocimiento de Haskell aunque no
sepan qué es la programación no determinista. Para ello, en este punto de introducción
explico de qué se trata este paradigma con lenguajes ya existentes como Curry. Después
explicaré cómo Sharade aborda este paradigma con la combinación de los dos tipos de semántica
que existen. El resto del documento trata sobre la implementación de Sharade, que debido a la
complejidad subyacente en la implementación de este paradigma en Haskell, es necesario
explicar bastantes conceptos de programación monádica en el punto \ref{sec:con_prev}. Después
comentaré la implementación de Fischer que soluciona el problema de la compartición con
evaluación perezosa en el apartado \ref{sec:explicit_sharing}.
En el punto \ref{sec:parsec} haré una breve explicación sobre la biblioteca \verb`Parsec` que
utilizo para realizar el análisis sintáctico de Sharade. Una vez el lector se sienta cómodo
con los conceptos de programación monádica, indeterminismo y las dos semánticas, abordaré la
implementación de Sharade, desde el análisis sintáctico, análisis de tipos y traducción. Por
último habrá un apartado de conclusiones, donde evaluaré el cumplimiento de los objetivos del
trabajo, el interés de la propuesta y algunos comentarios sobre Haskell. También recogeré
algunos comentarios sobre posibles trabajos futuros, como mejorar el rendimiento de Sharade
y la implementación del proyecto, cómo se pueden incorporar nuevas características al
lenguaje, como tipos de usuario y otras formas de recolectar resultados.

\end{document}
