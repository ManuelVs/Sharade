\documentclass[class=article, crop=false]{standalone}
\usepackage[utf8]{inputenc}
\usepackage[spanish]{babel}
\usepackage{minted}
\usepackage{xcolor}

\definecolor{bg}{rgb}{0.95,0.95,0.95}

\begin{document}

\section{Biblioteca parsec}\label{sec:parsec}
La biblioteca Parsec nos permite analizar sintácticamente un lenguaje mediante la creación
de funciones de orden superior de una manera muy expresiva. Se puede crear un analizador
sintáctico descendente sin la necesidad de crear un analizador léxico directamente. También
ofrece muchas primitivas muy útiles, como reconocer números, identificadores, especificar
nombres reservados, el formato de los comentarios, especificar la prioridad de los
operadores, tener en cuenta la indentación...

Parsec implementa todas estas operaciones haciendo uso de la mónada con estado
`\verb`ParsecT s u m a`'. Además, es una \verb`MonadTrans`. Es un tipo que permite un flujo
de caracteres \verb`s`, un tipo de estado de usuario, \verb`u`, una mónada \verb`m` y un
valor de retorno \verb`a`. Este valor de retorno serán las sucesivas partes del lenguaje, un
número, una expresión, un patrón... \verb`ParsecT` es una mónada con estado, donde el estado
es la porción de la entrada que queda por consumir. Las primitivas de la biblioteca son las
que se encargan de actualizar este estado `oculto'.

Vamos a dar algunos ejemplos de cómo sacar el máximo partido a las primitivas de la
biblioteca:

\begin{minted}[bgcolor=bg]{haskell}
import Text.Parsec.Indent
import qualified Text.Parsec.Token as Tok

langDef = Tok.LanguageDef
  { Tok.commentStart    = "{-"
  , Tok.commentEnd      = "-}"
  , Tok.commentLine     = "--"
  , Tok.nestedComments  = True
  , Tok.identStart      = letter
  , Tok.identLetter     = alphaNum <|> oneOf "_'"
  , Tok.opStart         = oneOf ":!#$%&*+./<=>?@\\^|-~"
  , Tok.opLetter        = oneOf ":!#$%&*+./<=>?@\\^|-~"
  , Tok.reservedNames   = ["choose", "let", "in", "case", "of"]
  , Tok.reservedOpNames = [",", "->", "\\", "=", "?", "&&", "||",
                           "<", "<=", ">", ">=", "==", "!=", "+",
                           "-", "*", "/"
                          ]
  , Tok.caseSensitive   = True
  }
lexer = Tok.makeTokenParser langDef

parens = Tok.parens lexer
reserved = Tok.reserved lexer
reservedOp = Tok.reservedOp lexer
identifier  = Tok.identifier lexer
integer = Tok.integer lexer

expression :: ParsecT String () Identity Expression
expression = do
  left <- identifier
  reservedOp "+"
  right <- identifier
  return (Add left right)

parseExpr :: String -> Either ParseError Expr
parseExpr s = runParser expression "" s
\end{minted}

Suponiendo que existiera el tipo \verb`Expr`, este ejemplo es capaz de reconocer y de crear
el árbol de expresión de secuencias como `a + b'. Las primitivas como \verb`identifier`,
\verb`reservedOp` y \verb`reserved` permiten reconocer identificadores, operadores reservados
y palabras reservadas tal como están descritas en el `analizador léxico'. También ofrece
primitivas algo más complejas como \verb`parens`, que dada una función que es capaz de
reconocer algo del tipo \verb`a`, es capaz de reconocer lo mismo pero rodeado de paréntesis.

Para permitir el backtracking en el análisis, por ejemplo de un lenguaje LL(0), existe la
primitiva \verb`try`, que intenta utilizar el analizador pasado por parámetro. Si ese
analizador tiene éxito, no hace nada, pero en caso de que falle, actualiza el estado `oculto'
y falla, para poder utilizar otro analizador.

El lenguaje también ofrece primitivas para operadores binarios e infijos de distinta
asociatividad. No hemos aprovechado estas primitivas en nuestra implementación debido a la
poca flexibilidad que ofrecían para generar expresiones.

Gracias a la gran flexibilidad del tipo \verb`ParsecT` se pueden crear gran cantidad de
extensiones para la biblioteca. Dos extensiones que hemos utilizado son \verb`parsec3-numbers`
y \verb`indents`. La extensión \verb`parsec3-numbers` permite analizar números enteros y
reales en gran cantidad de formatos, con más potencia que el analizador por defecto de
\verb`Parsec`. La extensión \verb`indents` es capaz de tener en cuenta la indentación, muy
útil para analizar ciertas expresiones, como `case of', un grupo de definiciones de funciones
o un conjunto de expresiones `choose' o `let'.

\end{document}
