\documentclass[class=article, crop=false]{standalone}
\usepackage[utf8]{inputenc}
\usepackage[spanish]{babel}

\begin{document}

\section{Resumen}

En este trabajo se propone e implementa un lenguaje funcional que incorpora características
habituales de este paradigma, como es el orden superior, ajuste de patrones o evaluación
perezosa. Añade características no tan habituales como es el indeterminismo, expresado
mediante funciones no deterministas, es decir, funciones que para unos argumentos dados
pueden devolver más de un resultado. Este tipo de funciones existen en los llamados lenguajes
lógico funcionales, que adoptan para el indeterminismo una semántica de compartición
(call-time choice), en contraste con la semántica de no compartición (run-time choice) más
típica de otros formalismos, como son los sistemas de reescritura.

Lo específico del lenguaje nuevo que se va a proponer e implementar, \textit{Sharade}, es que
combinará ambos tipos de semánticas mediante primitivas que existirán en el propio lenguaje.

La implementación del lenguaje está realizada íntegramente en Haskell, en contraste con las
implementaciones clásicas de los lenguajes lógico funcionales, habitualmente basadas en
Prolog. Haskell es un lenguaje funcional puro con un sistema de tipos muy fuerte, siendo el
lenguaje de referencia en el ámbito de la programación funcional con evaluación perezosa.
Aparte de fases auxiliares como son el análisis sintáctico y la inferencia de tipos, lo
esencial de la implementación consiste en un proceso de traducción de programas fuente
Sharade en programas objeto Haskell, aprovechando así muchas de las características de este
último como lenguaje funcional. Los programas objeto hacen uso intensivo de programación
monádica, a través en particular de una biblioteca Haskell para la programación con
indeterminismo con call-time choice y evaluación perezosa, biblioteca ya existente pero que
ha debido ser adaptada por nosotros para actualizarla a las nuevas versiones de Haskell.

\vspace{0.5cm}
\textbf{Palabras clave}

Programación Funcional, No determinismo, Curry, Toy, Sharade, Compartición, Call-time Choice,
Run-time Choice, Mónadas, Hindley-Milner

\end{document}
