\documentclass[class=article, crop=false]{standalone}
\usepackage[utf8]{inputenc}
\usepackage[spanish]{babel}
\usepackage{minted}
\usepackage{xcolor}
\usepackage{stmaryrd}

\definecolor{bg}{rgb}{0.95,0.95,0.95}

\begin{document}

\section{Call-time choice y evaluación perezosa}
Uno de los problemas que tiene la programación indeterminista en Haskell es que no cuenta con
construcciones que permitan la compartición de valores indeterministas en una expresión.
Recordemos que la semántica por defecto es la ya comentada `run-time choice'. Se podría
simular el comportamiento de la semántica `calltime choice' si se extrajesen los valores de
las mónadas para luego operar con ellos, pero esto plantea un inconveniente bastante
importante, se pierde la evaluación perezosa, ya que al extraer el valor de la mónada estamos
forzando su evaluación. Fischer en su trabajo propone una solución a este problema, que
comentaremos en el siguiente apartado. En el apartado \ref{sec:uso_directo} veremos algunos
ejemplos del uso de su biblioteca, donde finalmente quedará clara la necesidad de crear un
lenguaje para mostrar la combinación de semánticas.

\subsection{Biblioteca explicit-sharing}\label{sec:explicit_sharing}
Aunque el paradigma del no determinismo ya está completo en Haskell con las herramientas que
he explicado más arriba, en concreto con cualquier tipo que pertenezca a la clase
\verb`MonadPlus`, perdemos ciertas características, como la evaluación perezosa. Sin esta
característica, no podríamos trabajar con estructuras de datos infinitas o desechar cómputos
que que no nos interesen. Vamos a ver una serie de ejemplos para ilustrar esta pérdida de
características \cite{fischer2011purely}:

\begin{minted}[bgcolor=bg]{haskell}
import Control.Monad

-- Definición auxiliar
duplicate :: Monad m => m a -> m (a, a)
duplicate a = do
  u <- a
  v <- a
  return (u,v)
-- Definición auxiliar
const' :: a -> a -> a
const' = const

strict_bind :: (MonadPlus m) => m Int -> m (Int, Int)
strict_bind x = do
  v <- x
  duplicate (const' (return 2) (return v))

lazy_share x = do
  v <- share x
  duplicate (const' (return 2) v)
\end{minted}

Si pasamos un valor infinito o que se tarde mucho en calcular o directamente, provoque un
error de cómputo a \verb`strict_bind`, no se comportará correctamente, incluso cuando no se
está usando para nada el argumento. Esto es debido a la implementación del operador
\verb`>>=`, para cualquier tipo de mónada en la que se evalúe la función. En otras palabras,
perdemos la evaluación perezosa. Sin embargo, como veremos más adelante, esto no es así en
la otra función, \verb`lazy_share`. Con esa conseguimos la evaluación perezosa y sin perder
la importante característica de que todas las apariciones de \verb`v` correspondan con el
mismo valor.
\newpage

\subsubsection{Leyes de share}
La implementación del \verb`share` tiene que cumplir una serie de leyes para garantizar unas
características mínimas:
\begin{itemize}
  \item[-] Fail $\rightarrow$ \verb`share mzero = return mzero`. \\
  Compartir un cómputo vacío debe tener el mismo efecto. Es decir, se debe poder compartir un
  cómputo sin resultados sin que eso afecte al comportamiento del programa.
  \item[-] Ignore $\rightarrow$ \verb`share a >>= \x -> b = b`. \\
  Si la variable \verb`x` no aparece en la expresión de la lambda, el share se debe comportar
  como si no existiera. Esto garantiza el comportamiento explicado en el ejemplo anterior.
  Por ejemplo, si \verb`a` fuera un cómputo vacío o no terminante, el cómputo global no
  debería ser vacío o no terminante si no se usa ninguno de sus resultados.
  \item[-] Choice $\rightarrow$ \verb`share (a ? b) = (share a) ? (share b)`. \\
  Compartir un cómputo no determinista debe ser lo mismo que una elección de las
  comparticiones. Es decir, si una pieza del cómputo genera múltiples resultados, el
  resultado debe ser el mismo que si se coge cada uno de esos resultados, se evalúan en la
  expresión que contiene esa pieza y se juntasen todos los resultados nuevamente generados
  \cite{lopez2007simple}\cite{baader1999term}. Es una característica básica de la semántica
  `Call-time choice', por lo que es bastante deseable que esta primitiva, \verb`share`, la
  cumpla.
\end{itemize}

La tercera ley es fundamental para el correcto funcionamiento de la semántica `call-time
choice'. La diferencia entre las dos semánticas se aprecia bastante bien en términos
abstractos a través de una propiedad esencial que verifica `call-time choice' y no `run-time
choice'. Para formularla, introducimos una notación que explicamos vía ejemplos.
\begin{itemize}
  \item[-]  $\llbracket e\rrbracket$ $\rightarrow$ Es el conjunto de valores en el que se
            puede reducir la expresión \verb`e`. Por ejemplo,
            $\llbracket coin\rrbracket = \{ 0, 1\}$.
  \item[-]  $C[e]$ $\rightarrow$ Indica que la expresión \verb`e` aparece dentro de un
            contexto sintáctico $C[]$.
\end{itemize}

La propiedad fundamental de `call-time choice' es entonces: \\
$\llbracket C[e \: ? \: e']\rrbracket
=\llbracket C[e]\rrbracket \:\bigcup\: \llbracket C[e']\rrbracket
=\llbracket C[e] \:?\: C[e']\rrbracket$. \\
Con palabras, si una pieza en un contexto es
indeterminista, semánticamente debería ser equivalente a extraer esa pieza del contexto y
hacer la unión de los resultados incorporando cada uno de los valores de la pieza. Explicado
con un ejemplo puede quedar más claro: \\
$\llbracket duplicate \: (0 \: ? \: 1)\rrbracket = \{0, 2\}$ \\
$\llbracket duplicate \:0\rrbracket\:\bigcup\:\llbracket duplicate \: 1\rrbracket
=\{0\} \:\bigcup\: \{2\}$ \\
$\llbracket duplicate \:0 \:?\: duplicate \:1\rrbracket = \{0\} \:\bigcup\: \{2\}$ \\
Esta propiedad la describe F. J. López Fraguas en \cite{lopez2007simple}.

Las leyes Fail, Choice e Ignore hacen que se cumplan ciertos comportamientos en las
implementaciones que aseguran la evaluación perezosa y la compartición de nodos en una
expresión:

\begin{itemize}
  \item[-] Elección temprana $\rightarrow$ La elección de un valor se tiene que hacer como si
  se ejecutara de manera impaciente. Pero no se puede ejecutar impacientemente, por el
  ejemplo anterior, perderíamos la evaluación perezosa.
  \item[-] Demanda tardía $\rightarrow$ Se puede hacer una elección sin necesidad de efectuar
  el cómputo. Es decir, se elegirá el cómputo que hay que hacer pero no se efectuará hasta
  que sea realmente necesario, consiguiendo de nuevo la evaluación perezosa.
\end{itemize}
\newpage

\subsubsection{Simples pero malas implementaciones}
Para entender mejor estas leyes e intuiciones, vamos a ver dos formas incorrectas de
implementar \verb`share` y ver por qué no serían aptas.

\begin{minted}[bgcolor=bg]{haskell}  
import Control.Monad

duplicate :: Monad m => m a -> m (a, a)
duplicate a = do
  u <- a
  v <- a
  return (u,v)

coin :: MonadPlus m => m Int
coin = mplus (return 0) (return 1)

dup_coin_let :: MonadPlus m => m (Int, Int)
dup_coin_let = let x = coin in duplicate x

dup_coin_bind :: MonadPlus m => m (Int, Int)
dup_coin_bind = do
  x <- coin
  duplicate (return x)

dup_coin_share :: MonadPlus m => m (Int, Int)
dup_coin_share = do
  x <- share coin
  duplicate x

share :: MonadPlus m => m a -> m (m a)
share a = return a
\end{minted}

Esta implementación cumple las leyes `Fail' e `Ignore', pero no cumple la ley `Choice'. El
resultado de \verb`dup_coin_share` debería ser \{(0,0), (1,1)\}, pero tiene las cuatro
combinaciones. Esto es debido a que esta implementación comparte sólo el cómputo no
determinista, no los valores que genera. Así, en este caso \verb`dup_coin_share` es
equivalente a \verb`dup_coin_let`, le pasa el mismo cómputo no determinista.

La siguiente implementación de share sí cumple la ley de `Choice', pero no cumple `Ignore' y
`Fail', ya que hace una elección temprana del valor no determinista. Así, en el primer
ejemplo si se pasa como argumento un cómputo infinito a \verb`lazy_share` se convertiría en
otro cómputo infinito, a pesar de descartar el primer el argumento. Igualmente, si se pasa un
cómputo sin resultados, ¡la función no tendrá resultados! Este comportamiento sí que es
alarmante.

\begin{minted}[bgcolor=bg]{haskell} 
share :: Monad m => m a -> m (m a)
share a = a >>= \x -> return (return x)
\end{minted}

\subsubsection{Implementación sencilla}
Esta es una implementación sencilla que solo es capaz de compartir cómputos no deterministas
de enteros respetando las leyes. Este ejemplo sirve como toma de contacto para entender una
implementación más cercana a la real, que explicaré en el siguiente punto.

Para combinar la demanda tardía con la elección temprana del cómputo Fischer utiliza la
memoización. La idea es atrasar todo lo posible la elección del cómputo y recordar la misma
elección cuando se pide el valor real. Para ello, crea un tipo de datos que usa para recordar
si un valor ha sido evaluado o no:

\begin{minted}[bgcolor=bg]{haskell} 
data Thunk a = Uneval (Memo a) | Eval a
\end{minted}

También se crea un nuevo tipo, \verb`Memo`, que es el que se encargará de tener toda la lista
de los cómputos no deterministas, \verb`Thunk`s.

\begin{minted}[bgcolor=bg]{haskell} 
newtype Memo a = Memo {
  unMemo :: [Thunk Int] -> [(a, [Thunk Int])] }
\end{minted}

\verb`Memo` forma una mónada sobre \verb`a` y además se puede meter en la clase de tipos
\verb`MonadState` con \verb`[Thunk Int]` como estado. Es decir, nuestros programas no
deterministas tendrán un `estado oculto', que unido a la memoización, se encargará de
recordar las decisiones sobre la compartición. Veamos como se implementa todo esto:

\begin{minted}[bgcolor=bg]{haskell}
instance Functor Memo where
  fmap f m = Memo (\ts ->
    map (\(a, ts') -> (f a, ts')) (unMemo m ts))

instance Applicative Memo where
  pure = return
  f <*> m = Memo (\ts ->
  concatMap (\(a, ts') ->
    map (\(f, ts'') -> (f a, ts'')) (unMemo f ts'))
    (unMemo m ts))
\end{minted}

Las instancias de Functor y Applicative no son importantes, pero es necesario implementarlas
debido a que tenemos que meter \verb`Memo` en la clase de tipos \verb`Monad`.

Recordemos que \verb`fmap` recibía una función y el valor monádico. Tenemos que transformar
el valor metido en la `caja' aplicando la función. Pero esta vez, ese valor está en el
resultado de la función que tiene \verb`Memo`. Por lo tanto, aplicamos esa función,
\verb`unMemo m ts` y simplemente hacemos un map sobre la lista de parejas aplicando la
función al primer valor.

Applicative es la más complicada. Recibimos una función encerrada en un valor monádico y un
valor monádico. Tenemos que obtener el valor encerrado dentro del segundo valor, lo obtenemos
como antes. Después, usamos la lista de \verb`Thunk` que hemos obtenido para obtener la
función, que aplicamos a cada valor. Mientras tanto hemos obtenido otra lista de
\verb`Thunk`, que formará parte del resultado final.

\begin{minted}[bgcolor=bg]{haskell}
instance Monad Memo where
  return x = Memo (\ts -> [(x,ts)])
  m >>= f = Memo (\ts -> concatMap (\(x,ts) ->
    unMemo (f x) ts) (unMemo m ts))

instance Alternative Memo where
  empty = mzero
  (<|>) = mplus

instance MonadPlus Memo where
  mzero = Memo (\ts -> [])
  a `mplus` b = Memo (\ts -> unMemo a ts ++ unMemo b ts)

instance MonadState [Thunk Int] Memo where
  get = Memo (\ts -> [(ts,ts)])
  put ts = Memo (\_ -> [((),ts)])
\end{minted}

Para el operador \verb`>>=`, tenemos que sacar el valor encerrado aplicando la función a la
lista que recibimos. Después, como \verb`f` devuelve otro \verb`Memo`, usamos la lista
generada para obtener la lista final.

La implementación de \verb`Alternative` es idéntica a la de \verb`MonadPlus`, que simplemente
concatena las listas generadas.

\verb`MonadState` se parece mucho a los otros ejemplos. Con \verb`get` simplemente tenemos
que poner el valor como estado y con \verb`put` tenemos que poner el argumento que nos dan
como estado, además de limpiar el valor, poniendo \verb`()`.

\begin{minted}[bgcolor=bg]{haskell}
share :: Memo Int -> Memo (Memo Int)
share a = memo a

memo :: Memo Int -> Memo (Memo Int)
memo a = do
-- Obtenemos la lista
  thunks <- get
  let index = length thunks
  -- Metemos el nuevo valor al final
  put (thunks ++ [Uneval a])
  return $ do
    -- Obtenemos de nuevo la lista
    thunks <- get
    -- Probablemente haya cambiado, por eso es importante
    -- recordar (memoizar) la decisión tomada, `index`
    case thunks !! index of
      Eval x -> return x -- Devolver el valor
      Uneval a -> do
        -- Aquí se generaran realmente todos los valores
        -- posibles
        x <- a
        thunks <- get
        -- Sustituir el antiguo valor como valor "evaluado"
        let (xs,_:ys) = splitAt index thunks
        put (xs ++ [Eval x] ++ ys)
        -- Devolver el valor
        return x
\end{minted}

Creo que los comentarios entre las líneas del código ya son explicativos, pero aún así lo
explicaré con otras palabras. El estado `oculto' es una lista de valores que pueden estar sin
evaluar o evaluados, \verb`[Thunk Int]`. Con \verb`get` somos capaces de obtener esta lista.
Incorporamos el nuevo valor y guardamos su posición, \verb`index`. Devolvemos una `función'
(de 0 argumentos), que, cuando sea evaluada la primera vez (elección temprana), nos
encontraremos con que el valor en esa posición no está evaluado. Entonces se efectúa el
cómputo (evaluación tardía) y actualizamos la lista, depositándola de nuevo en el estado
`oculto'. Justo en la línea \verb`x <- a` surgirán todos los valores posibles. Esa línea
actúa como `fuente' de todos los resultados. Para entender el ejemplo de la fuente, un
ejemplo:
\begin{minted}[bgcolor=bg]{haskell}
f :: [Int]
f = do
  v <- [1,2,3,4]
  return (v + 1)
\end{minted}

La línea \verb`a <- [1,2,3,4]` actúa como fuente, \verb`v` irá tomando los valores y se
devolverán incrementados en uno. Los cómputos solo se efectúan cuando es necesario,
proporcionando una evaluación perezosa bastante sofisticada.

Como ya he dicho al principio de este punto, esta implementación respeta las tres leyes
básicas de las que habla Fischer, pero solo es capaz de compartir enteros. En el siguiente
punto explicaré una implementación muy parecida a la que usa Fischer en su biblioteca.

\subsubsection{Implementación real}
En la implementación real, Fischer declara el tipo \verb`Lazy`, que tendrá la misma función
que \verb`Memo` del ejemplo anterior. También usa un nuevo tipo, \verb`Store` que usará como
estado `oculto', que tendrá la misma función que la lista de \verb`Thunk Int`. Para
implementar \verb`Lazy` como instancia de \verb`MonadState` y las demás clases, delega en
otra clase de tipos algunas operaciones, \verb`Nondet`. Es decir, está implementado de tal
manera que cualquier tipo \verb`a`que pertenezca a esta clase, \verb`Lazy a` pertenecerá a
las clases de tipos \verb`MonadPlus`, \verb`MonadState`... En concreto, el tipo \verb`Set a`
está metido en esa clase, por lo que podremos observar casi cualquier cómputo en esta
estructura de datos. Vamos a ver las definiciones, esta vez no pondré la implementación
porque es bastante parecida a los ejemplos:

\begin{minted}[bgcolor=bg]{haskell}
class Nondet n where
  failure :: n
  (?)     :: n -> n -> n

newtype Lazy n a = Lazy {
    fromLazy :: (a -> Store -> n) -> Store -> n
  }

data Store = Store { nextLabel :: Int, heap :: M.IntMap Untyped }
\end{minted}

En esta implementación, \verb`Store` es como la lista de \verb`Thunk`, guarda los resultados
en un mapa y proporciona la claves para recordar la decisión. Fischer implementa funciones
auxiliares para actualizar y observar esta estructura, como inserciones, eliminaciones,
consultas... \verb`Lazy` es la mónada sobre la que se realizarán todas las operaciones.
Veamos la implementación del share:

\begin{minted}[bgcolor=bg]{haskell}
class Shareable m a where
  shareArgs :: Monad n =>
    (forall b . Shareable m b => m b -> n (m b)) -> a -> n a

class MonadPlus s => Sharing s where
  share :: Shareable s a => s a -> s (s a)

instance Nondet n => Sharing (Lazy n) where
  share a = memo (a >>= shareArgs share)

memo :: MonadState Store m => m a -> m (m a)
memo a = do
  key <- freshLabel
  return $ do
    thunk <- lookupValue key
    case thunk of
      Just x  -> return x
      Nothing -> do
        x <- a
        storeValue key x
        return x
\end{minted}

Lo primero es la existencia de una clase de tipos que se pueden compartir. Fischer mete en
esta clase casi todos los tipos primitivos de Haskell. Está pensada para compartir
recursivamente estructuras de datos más complejas, donde hay que compartir cada elemento y
además las sucesivas colas de la lista. Después está la clase de tipos \verb`Sharing`, donde
gracias a la implementación de Fischer, están todos los tipos \verb`Lazy a` si \verb`a`
pertenece a la clase de tipos \verb`Nondet`. La implementación del \verb`share` se basa en
compartir recursivamente el cómputo y después memoizarlo. La memoización es bastante parecida
a la de antes. Se obtiene una clave (antes era la longitud de la lista) y se devuelve una
función (de cero argumentos) que cuando sea evaluada, `recordará' la clave y el cómputo a
compartir. Si es la primera vez que se evalúa, se encontrará al consultar el estado `oculto'
que no está, generando los resultados e insertándolos. Si ya ha sido generado, tan solo hay
que devolverlo.

\subsection{Codificación directa}\label{sec:uso_directo}
Veamos unos ejemplos de cómo usar la biblioteca de Fischer para mostrar cómo se pueden
mezclar los dos tipos de semántica en Haskell sin perder evaluación perezosa:

\begin{minted}[bgcolor=bg]{haskell}
import Control.Monad
import Control.Monad.Sharing
import Data.Monadic.List

coin :: Sharing s => s Int
coin = (return 0) `mplus` (return 1)

duplicate :: Sharing s => s Int -> s Int
duplicate x = do
    u <- x
    v <- x
    return (u + v)

f :: Sharing s => s Int -> s Int -> s Int
f x y = do
  x' <- share x
  duplicate y

e1, e2, e3 :: Sharing s => s Int
e1 = duplicate coin
e2 = do
    c <- share coin
    duplicate c
e3 = f undefined coin  
\end{minted}

La primera definición se corresponde con \verb`coin = 0 ? 1` de Sharade. Como \verb`mplus`
opera con dos valores monádicos, es necesario meter el 0 y el 1 en sus respectivas mónadas
para luego juntarlas. La definición de \verb`duplicate` se corresponde con la definición en
Sharade de \verb`duplicate x = x + x`. Para conseguir las propiedades de `run-time choice' en
Haskell, es necesario `extraer' dos veces un valor del argumento. La función \verb`f` la
pongo de ejemplo para mostrar la evaluación perezosa y demostrar que la ley `Fail' e `Ignore'
funcionan en el tercer ejemplo.

El primer ejemplo llama a la función \verb`duplicate` directamente con \verb`coin`. Por lo
tanto, como el valor no está compartido, se generarán las cuatro posibilidades (dos
repetidas). El siguiente ejemplo muestra cómo conseguir la semántica `call-time choice'. Como
el valor de \verb`c` está ahora compartido, \verb`duplicate` solo generará dos resultados.
Por último, el tercer ejemplo muestra la evaluación perezosa. Como la función \verb`f` ignora
el primer argumento, el cómputo no da lugar a error, aunque se intente hacer una compartición
con él.

\begin{minted}[bgcolor=bg]{haskell}
import Control.Monad
import Control.Monad.Sharing
import Data.Monadic.List

mconcatenate :: Sharing s => s (List s a) -> s (List s a)
                             -> s (List s a)
mconcatenate mxs mys = do
    xs <- mxs
    case xs of
        Nil -> mys
        Cons x mxs -> cons x (mconcatenate mxs mys)

mreverse :: Sharing s => s (List s a) -> s (List s a)
mreverse mxs = mreverse' nil mxs where
    mreverse' rvs mxs = do
        xs <- mxs
        case xs of
            Nil -> rvs
            Cons mx mxs -> mreverse' (cons mx rvs) mxs

letter :: Sharing s => s Char
letter = (return 'a') `mplus` (return 'b') `mplus` (return 'c')

word :: Sharing s => s (List s Char)
word = nil `mplus` (cons letter word)

palindrome :: Sharing s => s (List s Char)
palindrome = do
    w <- share word
    let l = cons letter nil
    w `mconcatenate` (l `mplus` nil) `mconcatenate` (mreverse w)
\end{minted}

En este ejemplo las cosas se empiezan a complicar. Como todos los argumentos son monádicos,
no se puede trabajar con los valores de una manera directa y clara como se haría
tradicionalmente. Siempre hay que sacar los valores del interior de la mónada para
manipularlos, ensuciando mucho la sintaxis. Esta es una de las razones por las que propongo
Sharade, para poder trabajar con el indeterminismo de una manera fácil y limpia, con el
añadido de la combinación de las dos semánticas.

Las funciones \verb`mconcatenate` y \verb`mreverse` no tienen ningún tipo de indeterminismo,
pero tienen que tratar con la programación monádica. Podrían recibir un argumento
indeterminista, pero no están pensadas para esa situación. Lo primero que hay que hacer es
extraer el valor de la mónada y evaluarlo, distinguiendo entre el caso base y el caso
recursivo. Hay que mencionar que \verb`nil` y \verb`cons` son constructoras auxiliares del
tipo lista. Tienen la función añadida de meter el resultado en un valor monádico. La
definición de \verb`letter` es parecida a la de \verb`coin`, hay que meter cada valor en una
mónada y juntarlas. La definición de \verb`word` genera infinitos resultados indeterministas
de listas finitas e infinitas. Una palabra puede ser la lista vacía o una letra seguida de
otra palabra. En la definición de \verb`palindrome` se puede apreciar la compartición. Como
\verb`w` aparece dos veces en la expresión, es necesario que compartan el valor, si fueran
independientes \verb`palindrome` generaría todas las palabras posibles, es decir, no tendría
ninguna diferencia con \verb`word`. Para permitir los palíndromos impares la pieza intermedia
puede contener una letra o ser la lista vacía. El palíndromo se definiría en Sharade como
\begin{minted}[bgcolor=bg]{haskell}
palindrome = choose w = word in
             let l    = Cons letter Nil in
             mconcatenate w (mconcatenate (l ? Nil) w) ;
\end{minted}
habiendo definido antes las funciones auxiliares, claro.

Estos ejemplos muestran cómo es posible expresar directamente en Haskell la programación con
indeterminismo combinando `call-time choice' y `run-time choice'. También deja claro que esta
tarea en Haskell es complicada, propenso a errores y difícil de entender para nuevos
ingresados en la materia.

Sin embargo, como ya he dicho antes, Sharade es capaz de reducir esta complejidad,
descargando bastante la sintaxis y ofreciendo una abstracción de la programación monádica,
indeterminismo y combinando los dos tipos de semánticas muy potentes.

\end{document}
