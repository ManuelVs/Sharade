\documentclass[class=article, crop=false]{standalone}
\usepackage[utf8]{inputenc}
\usepackage[spanish]{babel}

\begin{document}

\section{Abstract}

This paper proposes and implements a functional language that incorporates common features of
this paradigm, such as higher order functions, pattern matching or lazy evaluation. It adds a
not so common feature, namely non-determinism, expressed through non-deterministic functions,
that is, functions that for given arguments can return more than one result. This type of
functions exist in the so-called functional logic languages, which adopt for indeterminism a
semantics of sharing (call-time choice), in contrast with the semantics of not sharing
(run-time choice) more typical of other formalisms, such as rewriting systems.

The specific thing about the new language that will be proposed and implemented,
\textit{Sharade}, is that it will combine both types of semantics through primitives that
exist in the language itself.

The implementation of the language is done entirely in Haskell, in contrast to the classic
implementations of functional logic languages, usually based on Prolog. Haskell is a pure
functional language with a very strong type system, being the reference language in the field
of functional programming with lazy evaluation. Apart from auxiliary phases such as syntactic
analysis and type inference, the essential part of the implementation consists of a
process of translating Sharade source programs into Haskell object programs, thus enabling
many of the latter's features as a functional language. The object programs make intensive
use of monadic programming, in particular through a Haskell library for programming with
indeterminism with call-time choice and lazy evaluation, library already existing but that
has had to be adapted by us to update it to new versions of Haskell.

\vspace{0.5cm}
\textbf{Keywords}

Functional programming, non determinism, Curry, Toy, Sharade, Sharing, Call-time Choice,
Run-time Choice, Monads, Hindley-Milner

\end{document}
